\section{理论基础}
\subsection{时间序列基础}
时间序列的定义,基本类型(平稳,周期,混沌),特性
\subsection{拓扑数据分析基础}
\subsubsection{单纯复形与滤过}
介绍如何从点云或函数构造单纯复形(主要是Vietoris-Rips复形和Čech复形),以及如何通过滤过来构造持久同调,解释滤过作为一种参数化地构建复形序列的过程
\subsubsection{同调与贝蒂数}
解释联通分支,环,空腔等同调群的直观意义和贝蒂数的计算
\subsubsection{持续同调}
追踪拓扑特征在滤过过程中"诞生"和"死亡"的生命周期
解释PH如何捕捉数据在不同尺度的结构信息
\subsubsection{持久性图与条码图}
定义PD和条码图作为PH的可视化表示
介绍PD的稳定性定理
后续可以添加简要的流程示意图
