\section{理论基础}

\subsection{时间序列}
时间序列是在一系列时间点上收集到的观测值的集合。更形式化地说,一个时间序列可以被定义为一个随机过程 $\left\{X_t\right\}$ 的一次观测实现(realization),其中 $t$ 属于某个索引集 $T$ 。在大多数应用中,索引集 $T$ 代表时间,并且通常被假定为离散且等间隔的,例如 $T=\{1,2, \ldots, N\}$ 或 $T=$ $\left\{t_1, t_2, \ldots, t_N\right\}$ ,其中 $N$ 是观测的总数。每个 $X_t$ 是在时间点 $t$ 观测到的数值。

\textbf{定义 2.1} (时间序列)\cite{shumway2000time}:
一个时间序列是一个按时间顺序索引的观测序列 $\left\{x_t: t \in T\right\}$ ,其中 $x_t$ 是在时间点 $t$ 记录的观测值,而 $T$ 是一个有序的时间索引集。

如果每个观测值 $x_t$ 是一个单一的数值(即 $x_t \in \mathbb{R}$ ),则该序列称为单变量时间序列(univariate time series)。如果每个观测值 $x_t$ 是一个 $k$ 维向量(即 $x_t \in \mathbb{R}^k, k>1$ ),则该序列称为多变量时间序列 (multivariate time series)。在本研究中,我们主要关注单变量时间序列。

时间序列分析的主要目标是理解这些观测序列的潜在结构和动态特性,以便进行描述、解释、预测或控制。这些序列可以来源于各种领域,如经济学(例如,股票价格)、气象学(例如,每日温度)、医学(例如,心电图信号)和工程学(例如,传感器读数)。

\subsection{点云}
在几何处理和数据分析领域,点云是描述和表示三维形状或高维数据对象的一种基本形式。点云的定义在理论文献中通常与拓扑和几何分析相关联,它构成了从离散采样数据中推断连续结构的基础。

\textbf{定义 2.2} (点云)\cite{dey2022computational}:
一个点云 $P$ 是在某个度量空间 $\left(M, d_M\right)$ 中的一个有限点集,通常我们考虑的是 $m$ 维欧几里得空间 $\mathbb{R}^m$ 及其标准欧几里得距离。因此,一个点云可以表示为 $P=\left\{p_1, p_2, \ldots, p_k\right\} \subset \mathbb{R}^m$ ,其中每个 $p_i$ 是一个 $m$ 维向量,代表空间中的一个采样点。

从计算拓扑的角度来看,点云通常被视为从某个未知的、潜在的几何对象或数据分布中采样得到的结果。拓扑数据分析(TDA)的核心任务之一便是从这些离散的点云数据中推断出该潜在对象的拓扑不变量和几何特征。点云本身不包含点与点之间的连接信息,但其空间分布蕴含了原始形状的结构。后续的拓扑分析方法,如构建单纯复形(例如 Vietoris-Rips 复形或 Čech复形),正是基于点云中点之间的邻近关系来显式地构建这些连接,从而揭示数据的"形状"。

在本研究中,时间序列数据将通过特定的嵌入技术(详见第3章)转换为高维欧几里得空间中的点云,这些点云随后将作为持续同调分析的输入,以提取其拓扑特征用于分类任务。

\subsection{持续同调的基本概念}
\subsubsection{单纯形与单纯复形}
\subsubsection{过滤}
\subsubsection{同调群与贝蒂数}
\subsubsection{持续同调}
\subsubsection{持续图}
