\section{从拓扑不变量中提取有效特征}
\subsection{引言:持久性图的挑战与向量化的必要性}
PD是多集合或图,不能直接输入标准的机器学习模型
需要将PD转化为固定唯独的向量,但这个过程可能损失信息
所以要介绍本章的不同方法,他们旨在以稳定且具有区分度的方式向量化PD

\subsection{基于持久性图统计量的特征}
\subsubsection{方法原理}
\subsubsection{特征示例}
点的数量,总/平均/最大/最小生命周期
持久性熵,基于PD点的各种范数
\subsubsection{优缺点}

\subsection{贝蒂曲线或序列}
\subsubsection{方法原理}
\subsubsection{文献实例}
\subsubsection{优缺点}

\subsection{持久性景观}
\subsubsection{方法原理}
\subsubsection{数学性质}
\subsubsection{文献实例}
\subsubsection{优缺点}

\subsection{持久性图像}
\subsubsection{方法原理}
\subsubsection{关键参数}
坐标变换,权重函数,核函数类型与参数,图像分辨率
分析这些参数对结果的影响
\subsubsection{优缺点}
\subsubsection{文献实例}

\subsection{持久性曲线}
\subsubsection{方法原理}
\subsubsection{文献实例}
\subsubsection{优缺点}

\subsection{持久性中心}
\subsubsection{方法原理}
\subsubsection{文献实例}
\subsubsection{优缺点}

\subsection{方法比较和深入讨论}
