\pagenumbering{arabic}
\section{绪论}
\subsection{研究背景与意义}
在数据挖掘和机器学习中,时间序列的早期分类受到了极大的关注,因为它可以解决包括医疗,工业和运输在内的许多领域的时间关键问题。
文献表明时间序列的早期分类有着很多的应用,以下进行详细讨论:

\begin{itemize}
    \item \textbf{行为分类}: 随着智能手机和可穿戴设备中多模态传感器的可用性,人们可以轻松监控他们的日常活动,
    如步行,跑步,饮食等。人类活动的早期分类有助于最大限度地减少系统的响应时间,
    从而改善用户体验[8]。[8]中的研究人员试图使用MTS生成的传感器对各种复杂的人类活动进行分类,
    例如坐在沙发上、坐在地板上、站着说话、上楼和吃饭。[37]-[39]中的研究集中在识别人类动作,如捡起,小鸡舞,高尔夫挥杆等
    \item \textbf{医疗诊断}:[3]、[43]-[46]、[46]-[51]工作的主要动机是开发哮喘、病毒感染、异常心电图等疾病的早期医学诊断分类方法。
    这些疾病的早期诊断可以显着减少对患者健康的影响并协助医生进行治疗。基因表达已用于研究患者的病毒感染、
    疾病的药物反应和患者恢复[43]-[45]。早期发现哮喘有助于预防危及生命的风险,并进一步提供快速缓解[49]。
    [51]中的研究侧重于使用体温和呼吸频率等生理测量的MTS来预测将患者转移到重症监护室(ICU)的正确时间。
    此外,ECG也是由心脏活动产生的电信号的时间序列。ECG的早期分类[3],[46],[47],[52]有助于最早诊断心脏跳动异常,降低心力衰竭的风险。
    \item \textbf{工业过程监控}:随着传感器技术的进步,通过使用传感器监测工业过程变得方便和轻松。传感器产生时间序列,该时间序列被分类以了解操作的状态。[29],[35],[42],[53]-[56]中的作者致力于通过使用传感器数据来构建基于早期分类的工业问题解决方案。在化学工业中,即使是轻微的泄漏也会对船员的健康造成危险影响[29]。早期分类不仅降低了健康风险,还通过确保始终平稳运行来最大限度地降低维护成本。特别地,在[29]中使用气体传感器开发了电子鼻,以闻到气体气味。它会生成MTS,需要尽早对其进行分类,以检测任何泄漏。在[42]中,作者试图检测液压系统中的泵泄漏、压力降低和操作效率低下等问题。及早发现这些问题可以显著降低维护成本。核电厂仪表故障的早期识别可以避免危险后果[35],[56],[57]。

\end{itemize}

为应对这一需求,研究者不断探索新型特征表示方法,其中基于拓扑数据分析(Topological Data Analysis, TDA)的持续同调技术正逐渐成为突破传统方法瓶颈的重要方向。其基本思想在于利用持续同调捕捉时间序列中的拓扑不变性特征,进而揭示数据背后隐藏的全局结构规律。与传统统计或深度学习方法相比,基于持续同调的方法具有多项显著优势:

\begin{itemize}
    \item 强鲁棒性: 拓扑特征对噪声和尺度变化不敏感,能够有效克服局部波动的干扰。例如,在机械振动信号中,持续同调能够稳定识别出轴承早期故障所对应的环状拓扑模式,而传统的频域分析则容易受到随机噪声的影响。
    \item 结构解释性强: 通过持续同调条形码,可以直观地呈现数据中的关键拓扑模式(如 H1 环状结构和高维空洞),从而增强分类过程的可解释性。在医疗研究中,癫痫 EEG 信号中的短暂高频振荡现象,往往可通过 H1 条形码的突然延长被精准捕获。
    \item 多尺度分析能力: 结合时间延迟嵌入与子窗口分割技术,该方法可同时捕捉长期趋势与瞬时动态。例如,加州大学团队提出的滑动窗口持续同调方法成功检测出了电力负荷序列中的周期性异常。
\end{itemize}

综上所述,基于持续同调的时间序列分类技术对实际应用具有重要意义。它不仅能够帮助工业设备在复杂工况下实现故障早期预警,如在风力发电机振动监测中识别叶片裂纹;在医疗领域,该技术能够辅助诊断阿尔茨海默症患者的脑电信号异常;在智慧城市建设中,还能通过分析交通流量序列的周期性拥堵特征来优化路网规划。未来,随着拓扑数据分析理论和边缘计算技术的不断进步,该方法将推动工业物联网、精准医疗等领域的智能化升级,为复杂系统的决策提供更加可靠的理论支撑。



\subsection{拓扑数据分析简介及其应用于时间序列的动机}
在时间序列分类领域,研究者们提出了多种方法来提高分类性能。以下是一些主要的研究方向和方法:基于形状的方法,基于结构的方法,基于区间的方法,基于集成的方法,基于深度学习的方法.

基于形状的方法主要在于寻找能够区分不同类别的局部模式和子序列。形状方法试图从时间序列中提取局部具有判别性的形状特征,这些特征往往可以直接反映数据中关键的模式变化,从而实现分类。例如,通过比较时间序列中是否存在某个特定的形状(如上升或下降趋势、周期性波动等)来判断类别。
结构方法更侧重于捕捉时间序列的全局或整体特性,如整体趋势、周期性、平稳性等。它们可能通过建立全局模型(如自回归模型、状态空间模型)来描述数据的内在结构,或者利用全局距离度量(如DTW)比较序列的相似性。这类方法强调数据整体形态的结构信息而非局部细节。
区间方法则通过从时间序列中划分出若干个子区间,分别提取区间内的统计特征(如平均值、方差、峰值等),然后利用这些特征构建分类器。这种方法的优势在于可以捕捉到不同时间段内表现出不同统计规律的信息,尤其适用于序列中存在明显局部变化或阶段性特征的情况。
集成方法综合了多种单一分类器或特征提取方法的优势,通过组合多个模型(例如集成形状、区间和结构信息的多个模型),来提高整体分类准确率和鲁棒性。这种方法通常能够抵消单一模型的偏差,并在面对复杂多样的数据时展现出更好的泛化能力。
深度学习方法通过构建多层神经网络(如CNN、RNN、Transformer等),自动从原始数据中学习特征表示,无需人工设计特征。这类方法在大规模数据和复杂模式识别任务中表现突出,能够捕捉到数据中的非线性关系和多尺度信息,但同时也需要更多的数据和计算资源。下面我将介绍一下国内外的研究现状。

Bagnalla 等人 (2017) 对当前时间序列分类方法进行了系统性的综述和广泛的实验评估,涵盖了来自 UCR 存档的 85 个数据集上的 18 种先进算法。他们的研究不仅表明了简单的基线方法(如基于 DTW 的 1-最近邻方法)在很多情况下依然具有竞争力,而且强调了结合多种表示方式的集成方法在性能上具有显著优势。该工作为国内外时间序列分类的研究提供了重要的参考依据,推动了该领域标准化评估方法的建立,也为后续算法的改进和新方法的探索奠定了坚实的实验基础。
Chen 和 Ng (2004) 在他们的论文《On the marriage of Lp-norms and edit distance》中提出了一种将Lp范数和编辑距离相结合的统一框架,详细探讨了如何利用两者的优点来度量序列之间的相似性。他们的研究展示了通过结合连续的Lp距离度量与离散的编辑操作,可以更准确地反映字符串或序列数据的结构特征,为大规模数据检索和序列比较提供了理论支持。该工作在国际数据库大会上发表,为后续在文本、DNA序列等领域的相似性分析方法研究提供了重要的参考。
Marteaup (2009) 提出了一种改进的时间弯曲编辑距离方法,通过引入刚性调节参数来平衡匹配过程中的灵活性和约束性,从而更好地适应时间序列数据中的局部形变。该方法在传统编辑距离的基础上,结合了动态时间规整的思想,能够有效处理因时间扭曲、局部缩放等因素引起的匹配误差问题。实验结果表明,这种方法在模式匹配和相似性搜索任务中不仅提高了匹配准确率,同时保持了较低的计算复杂度,为时间序列匹配领域提供了一种具有较高实用价值的解决方案。
Stefana、Athitsos 和 Dasg (2013) 在他们发表于 IEEE Transactions on Knowledge and Data Engineering 的论文中提出了 Move-Split-Merge (MSM) 距离度量方法。该方法通过引入“移动”、“分割”和“合并”三种基本操作,扩展了传统编辑距离的思想,从而更灵活地处理时间序列数据中的局部非线性变形问题。相比于传统的动态时间规整(DTW)等距离度量,MSM 能够在保留序列局部结构信息的同时,更准确地衡量序列之间的相似性。实验结果表明,MSM 在时间序列分类和相似性搜索任务上表现出更高的鲁棒性和准确性,为时间序列数据挖掘提供了一种有效的新工具。
Schäfer (2015) 在其发表于 Data Mining and Knowledge Discovery 的论文中提出了 BOSS 方法,专注于解决噪声环境下的时间序列分类问题。该方法通过对时间序列数据进行符号化处理,有效地提取出对分类任务具有判别力的特征,同时增强了对噪声的鲁棒性。实验结果表明,在存在较高噪声干扰的情况下,BOSS 能够显著提升分类准确率,为噪声环境下的时间序列分析提供了一个高效且稳健的解决方案。
Schäfer 和 Leser(2017)在他们的论文中提出了 WEASEL 方法,这是一种基于模式袋(bag-of-patterns)思想的快速且准确的时间序列分类方法。该方法利用滑动窗口、傅里叶变换以及统计特征选择相结合的策略,将原始时间序列转换为简洁且具有判别力的符号特征向量,从而实现高效的分类。大量在标准数据集上的实验表明,WEASEL 不仅在分类精度上具有竞争力,而且计算复杂度远低于传统方法,使其在大规模时间序列分类任务中具有较高的实用价值。该研究成果发表于 2017 年 ACM 信息与知识管理会议,为时间序列分类领域提供了一种有效且高效的解决方案。
Morrill 等人(2023)提出了一种基于广义签名(generalised signature)的多变量时间序列特征提取方法。该方法扩展了传统签名变换,旨在捕捉多变量时间序列中各维度之间的交互和时序信息,从而生成具有丰富描述力的特征表示。这篇预印本在理论上探讨了广义签名的数学性质,并通过一系列实验验证了其在下游任务(如分类和回归)中的鲁棒性和有效性。该工作为处理高维、多变量时间序列数据提供了一种新颖且具有竞争力的特征提取工具。
Cabellon 等人提出了一种基于监督区间搜索的时间序列分类方法,旨在提高分类的速度和准确性。他们的方法核心在于自动识别时间序列中最具判别力的区间,并利用监督信号来优化区间选择过程,而不是直接对整个序列进行分析。这样既能捕捉到关键局部特征,又能大幅降低计算开销。实验结果表明,该方法在多个标准数据集上均取得了与最先进算法相当甚至更优的分类效果,同时在运行效率上也有明显优势,为实际应用提供了有力支持。
lynn、Large 和 Bagnall(2019)在论文中提出了 Contract Random Interval Spectral Ensemble (c-RISE) 方法,该方法探讨了在构建集成分类器过程中对区间进行“压缩”处理对分类准确率的影响。具体来说,c-RISE 方法通过随机区间划分和频谱特征提取,将原始时间序列划分为多个局部区间,并对这些区间进行特征压缩,进而构建一个多样化且高效的集成分类器。实验结果表明,采用这种压缩策略不仅能降低模型复杂度和计算成本,同时在多数数据集上提升了分类准确性和鲁棒性。该研究为时间序列分类中如何在保证关键信息的同时减小模型规模提供了新的思路,并为后续相关方法的发展奠定了理论和实践基础。
Middlehurst 等人(2021)在《Machine Learning》期刊上发表的论文中提出了 HIVE-COTE-2.0,一种新型的元集成方法,用于时间序列分类。该方法融合了多种不同类型的分类器(例如基于距离、基于区间、基于形状等),构建了一个统一的分类框架,从而在准确性和鲁棒性方面实现了显著提升。与前一代 HIVE-COTE 方法相比,HIVE-COTE-2.0 不仅在实验中展示了更高的分类准确率,同时也在计算效率上有所改善。这项工作为时间序列分类领域提供了一个创新的、具有竞争力的解决方案,并为未来元集成方法的进一步研究奠定了坚实的理论和实践基础。
Shifaz 等人(2020)在他们发表于 Data Mining and Knowledge Discovery 的论文中提出了 TS-CHIEF 算法,这是一种针对时间序列分类的可扩展且准确的森林算法。该方法通过构建多棵决策树的集成,利用随机特征抽样和专门设计的时间序列转换技术,能够捕捉到时间序列中的关键模式和局部特征。TS-CHIEF 在保持较高分类准确率的同时,也显著降低了计算复杂度,使其能够高效处理大规模数据集。实验结果显示,该算法在多个标准数据集上的表现均优于或相当于现有最先进的时间序列分类方法,为实际应用和后续研究提供了一种高效、稳健的解决方案。
Wang、Yan 和 Oates(2017)在论文中提出了一种完全从零开始训练的深度神经网络,用于时间序列分类,并证明这一简单架构能够作为一个强有力的基线方法。他们的方法无需手工设计特征,而是直接从原始时间序列数据中自动学习判别特征。实验结果显示,该方法在多个标准数据集上均表现出与当前复杂方法相当甚至更优的分类性能,同时具有较高的效率,为深度学习在时间序列分类领域的应用提供了坚实的基础。
这篇论文提出了 TimesNet,一种全新的时间序列分析模型,其核心思想在于将传统的一维时间序列数据转换为二维表示,从而捕捉时间序列中的二维变化特征。论文中作者通过构造二维卷积模块来建模序列数据的全局与局部动态变化,并在多项任务上展示了该模型卓越的性能和泛化能力。实验结果表明,TimesNet 在处理各种时间序列分析任务时,不仅能够准确捕捉时序中的变化信息,还能有效提升模型的预测和分类准确率。
该论文以全球主要股市为研究对象,基于拓扑结构方法构建了全球股市联动网络,并探讨了市场间的联动关系如何影响系统性金融风险的传递与聚集。作者通过实证分析揭示了不同股市之间的结构性联系及其在危机时刻的传染机制,指出拓扑结构在捕捉市场风险传染路径和识别金融系统脆弱性方面具有重要作用。这项研究不仅为理解全球股市动态和风险传播提供了新的视角,也为监管机构制定防范系统性风险的策略提供了理论支持。
该论文提出了一种基于拓扑数据分析的时间序列分类方法。作者利用持续同调等拓扑工具对时间序列数据进行特征提取,通过构造拓扑图谱(如条形码等)将原始数据的全局形态信息映射到低维特征空间,从而实现对时间序列的高效分类。实验结果显示,这种方法不仅能有效捕捉数据中的结构性和不变性特征,而且在面对噪声干扰时表现出较强的鲁棒性,为时间序列分类任务提供了一种全新的分析视角和实用工具。
Karan A 和 Kaygun A (2021) 在《Expert Systems with Applications》期刊中提出了一种基于拓扑数据分析的时间序列分类方法。该方法利用持续同调等拓扑工具提取时间序列中的不变性和全局结构特征,然后将这些拓扑特征作为输入送入传统的机器学习分类器进行判别。实验结果表明,该方法在捕捉时间序列中关键形态信息以及抵抗噪声干扰方面具有较高的准确率和鲁棒性,为时间序列分类提供了一种全新的分析视角和有效工具。
海彤(2021)在《基于拓扑分析的时间序列分类》这篇学位论文中,探讨了如何利用拓扑数据分析方法来处理和分类时间序列数据。论文首先构建了时间序列的拓扑表示,如利用持续同调生成条形码,从而提取数据中的全局结构和不变性特征。接着,作者将这些拓扑特征与传统的机器学习分类器相结合,验证了其在提高分类准确性和鲁棒性方面的潜力。实验结果表明,基于拓扑分析的方法能够捕捉到时间序列数据中难以通过传统统计特征直接反映的重要信息,为时间序列分类提供了一种新颖而有效的研究思路。这项研究不仅丰富了时间序列分析的理论体系,也为实际应用中的风险评估和异常检测等问题提供了有力支持。

\subsection{研究内容与目标}

通过国内外研究现状可以发现,在时间序列的分类中,有些研究利用传统的统计特征进行分类,有些研究利用深度学习的方法进行分类,还有一些研究利用拓扑数据分析的方法进行分类。传统的统计特征方法在处理复杂数据时往往难以捕捉到数据中的非线性关系和多尺度信息,而深度学习方法虽然具有强大的特征学习能力,但在小样本或高维数据上可能面临过拟合的问题。拓扑数据分析方法则通过捕捉数据的全局结构和不变性特征,能够有效克服这些问题。因此,本文将重点研究基于拓扑数据分析的时间序列分类方法,并结合深度学习技术进行改进和优化。


















