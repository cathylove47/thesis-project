\pagenumbering{arabic}
\section{绪论}
\subsection{研究背景与意义}
随着信息技术的飞速发展和数据采集与处理能力的显著增强,时间序列数据已成为描述自然现象、社会经济活动及工程系统动态过程的核心信息载体。对这些蕴含丰富动态信息的时间序列数据进行有效分类,即时间序列分类(Time Series Classification, TSC),对于深入理解复杂系统行为、准确预测未来趋势以及制定科学的智能决策均具有至关重要的理论与实践意义。

时间序列分类的应用价值已在诸多领域得到广泛验证与体现。例如,在行为识别领域,通过分析来自智能设备的多模态传感器数据,对用户的日常活动(如步行、跑步及特定手势等)进行精确分类,不仅有助于提升用户体验,更能实现即时的智能响应。在医疗健康领域,对心电图(ECG)信号、脑电图(EEG)信号或基因表达谱等生物医学时间序列数据进行早期分类与模式识别,能够为多种疾病(如心脏功能异常、哮喘发作、病毒感染以及神经退行性疾病等)的辅助诊断、风险预警乃至个性化治疗方案的制定提供有力支持。在工业过程监控与智能运维领域,通过对各类传感器产生的海量时间序列数据进行深度分析,可以实现对设备潜在故障(例如泵机泄漏、仪表失灵等)的早期预警,并准确预测维护需求,从而有效降低运营风险、优化资源配置并提升生产效率。值得一提的是,许多研究已经开始探索利用持续同调等先进技术分析生理信号,例如,Umeda等人将持续同调应用于肌电图(EMG)信号的活动分类 \cite{umeda2017time},Majumder等人利用EEG信号检测自闭症谱系障碍 \cite{majumder2020detecting},Altındiş等人研究了EEG信号的最佳延迟嵌入参数选择 \cite{altindics2021parameter},Ignacio等人将其用于ECG信号的分类 \cite{ignacio2019classification},以及Erden和Cetin将其应用于呼吸速率的估计 \cite{erden2017period}。特别是在众多对实时性要求极高的应用场景中,如前述的疾病早期预警系统或工业故障即时诊断系统,时间序列的早期分类能力更是显得尤为关键和核心。

尽管目前已存在众多时间序列分类方法,涵盖了基于统计特征的方法、基于距离度量的方法(如动态时间规整DTW \cite{rakthanmanon2012searching})、基于频域分析的方法以及近年来迅速发展的深度学习模型等 \cite{altun2010human,1024896851.nh},然而,这些传统与新兴方法在处理具有复杂非线性动力学特性、显著多尺度结构特征或受到严重噪声污染的时间序列数据时,仍普遍面临一系列挑战。这些挑战主要包括特征提取不充分、模型解释性不足以及对原始数据质量要求较高等方面。

具体而言,传统的统计特征往往难以有效捕捉数据中隐含的非线性关系和全局结构信息,其表征能力在高维复杂时间序列面前显得捉襟见肘。另一方面,深度学习方法虽然凭借其强大的自主特征学习能力在某些任务上取得了显著成功,但在处理小样本数据或高维稀疏数据时,容易面临过拟合的风险,从而影响模型的泛化性能。更为重要的是,深度学习模型固有的“黑箱”特性,往往使得其决策过程难以理解和追溯,这在许多要求高可解释性的关键领域(如医疗诊断、金融风控)中受到了限制。这些固有局限性极大地驱动着学术界和工业界的研究者们持续不断地探索全新的理论框架与技术方法,以期能够更深刻、更全面地理解和表征复杂时间序列数据,进而提升分类任务的性能与可靠性。

在此背景下,拓扑数据分析(Topological Data Analysis, TDA)\cite{JSJC20250411004}作为一种新兴的数学理论与数据分析工具,因其能够从复杂高维数据中提取稳健的、关于数据“形状”的全局和局部拓扑不变量,为时间序列分类研究提供了全新的视角和强大的分析手段~。持续同调(Persistent Homology, PH)作为TDA的核心技术,通过量化拓扑特征在不同尺度下的演化,能够捕捉到传统方法难以揭示的时间序列内在结构模式~。基于持续同调的方法通常具有较强的鲁棒性,对噪声和度量扰动不敏感,能够有效克服局部波动的干扰,并且其输出(如持续图或条形码)为理解数据的结构提供了可解释的途径~。例如,在机械振动信号中,持续同调能够稳定识别出轴承早期故障所对应的环状拓扑模式;在医疗研究中,癫痫EEG信号中的短暂高频振荡现象也可能通过持续同调的H1维特征被精准捕获~。因此,深入研究并发展基于持续同调的时间序列分类算法,对于提升复杂时间序列分析的性能、增强模型的可解释性,具有重要的理论价值和广阔的应用前景~。本研究旨在系统梳理并探讨此类算法的关键技术与发展方向。我们的拓扑分析的基础在于持久同源性理论,它允许持久图的构造。这样的图表可以被看作是概括性统计,其捕获多尺度拓扑特征的报告\cite{fasy2014confidence}。同源性可以检测诸如连接的组件、隧道和空隙等特征。持续图允许人们研究当缩放参数变化时这些特征持续多长时间。因此,持续时间较长的特征被认为是最显著的特征。例如,考虑二维圆的点云将在持续图中产生长期持续的二维洞。


\subsection{国内外研究现状}
拓扑数据分析(TDA),特别是其核心工具持续同调(PH),近年来在时间序列分析与分类领域受到了国内外学者的广泛关注与积极探索。研究表明,TDA能够有效地捕捉数据的多尺度拓扑结构,为理解复杂时间序列提供了新的维度。这些研究大致可以分为两大方向:一是基于时间延迟嵌入(Time Delay Embedding, TDE)将时间序列转化为点云数据进行拓扑分析,二是基于图表示(Graph Representation)构建时间序列的拓扑结构~。

\textbf{国际研究现状}

国际上,学者们在多个领域成功应用PH进行时间序列分析。一个显著的趋势是将TDA与传统机器学习方法相结合,构建混合模型以提升分析性能~。例如,在海洋学数据分析中,通过提取时间序列的拓扑特征并将其融入支持向量回归(SVR)和层次聚类分析(HCA)等模型,显著提高了预测精度和聚类效果~。\cite{lin2025hybridizing}研究者发现,持续同调不仅能够揭示数据的“拓扑信号”(长条码),其短条码同样蕴含着重要的几何信息,如点云采样自的曲面的曲率等~。

针对时间序列数据的多样性,有研究致力于开发能够融入领域知识的PH计算方法。\cite{2}例如,通过调整影响向量来定制化PH分析过程,并证明了此类方法的稳定性,即对影响向量的调整具有稳健性~。这种方法使得基于图表示的时间序列拓扑分析能够更好地适应特定领域的应用需求~。

在时间序列异常检测方面,基于PH的算法也显示出潜力。一些研究利用PH中的“环”结构检测来识别多种类型的异常\cite{boispersistent},如突变、尖峰以及重复性或非重复性异常,并提供了理论保证其能够近似识别时间序列中的“正常”部分~。这与传统的基于滑动窗口比较持续图的方法不同,后者可能在处理重复异常或多种正常模式时产生假阴性或假阳性,且计算成本较高~。

将持续图信息转化为适用于机器学习算法的向量表示是TDA领域的一个核心研究方向。“持续性曲线”(Persistence Curves)便是一种将持续图信息规范且灵活地编码为向量的方法,其稳定性也得到了理论证明\cite{chung2020persistent},并在UCR时间序列分类基准数据集上取得了优于相关基准的性能~。此外,还有研究者探索了基于持续性图像(Persistence Images)的轨迹分类方法,通过将持续图转换为图像表示,再利用逻辑回归等分类器进行分类~。\cite{adams2017persistence}

在生物医学信号分析领域,PH的应用也十分活跃。例如,有研究针对功能性磁共振成像(fMRI)时间序列数据开发了基于PH的统计推断方法,通过多层块采样蒙特卡罗检验克服了传统方法对数据独立性假设的依赖,成功检测了任务相关的拓扑组织结构~。\cite{abdallah2023statistical}Perea和Harer\cite{perea2015sliding}通过滑动窗口和持续同调分析信号的周期性,为信号分析提供了拓扑视角~。Karan和Kaygun\cite{3}则利用持续同调从生理信号中提取拓扑特征进行应激识别~。

\textbf{国内研究现状}

国内学者在基于TDA的时间序列分类领域也取得了一系列进展,积极探索其理论创新与应用拓展。
严银凯等人\cite{JSJC202406009}提出的“基于持续同调的倾斜时间序列分类算法”(PHTSI)是一个代表性的工作~。该算法针对现有时间序列分类算法在高维拓扑信息及时序顺序信息提取能力不足的问题,创新性地结合时序数据方差将单变量时间序列嵌入二维点云,以展现周期内和周期间的时序变化。通过在滑动窗口划分的子区间上进行“时间倾斜”操作,将点云分解为不同结构,从而增强算法对多样化时序数据的适应性,并有效捕捉时序顺序信息。该算法利用持续同调技术在点云上构建Vietoris-Rips (VR)复形流,分析不同尺度下各维度孔洞数量的变化,以提取更全面的拓扑结构特征。最终,通过计算持久性中心表示向量作为输入,使用随机森林模型进行分类。在多个UCR时间序列数据集上的对比实验表明,PHTSI算法在分类精度和F1值方面均取得了显著提升~。这项工作体现了国内研究在优化点云构建策略和特征提取方法方面的努力。

海彤\cite{1021736289.nh}在其学位论文中也对基于拓扑分析的时间序列分类方法进行了探讨~,反映了TDA方法在国内学术界的逐步渗透。

此外,国内研究者还将TDA应用于具体的实际问题中。例如,刘雨诗等人\cite{刘语诗2022基于拓扑非线性动态建模的神经退行性疾病异常步态识别}将持续同调应用于神经退行性疾病患者的异常步态识别。通过将步态波动时间序列利用延迟嵌入重构为相空间点云,再利用持续同调提取点云的拓扑描述信息,构建拓扑非线性动态特征,并结合机器学习模型进行分类,取得了良好的识别效果~。这表明TDA在国内生物医学信号处理领域的应用潜力。张李轩等人\cite{张李轩2021基于}则基于MEMS传感器数据的拓扑分析进行了用户识别研究~,拓展了TDA在身份认证等领域的应用。

\textbf{研究挑战与方向}
尽管国内外研究均取得了显著进展,但基于持续同调的时间序列分类仍面临一些挑战与值得深入探讨的方向:
\begin{itemize}
    \item \textbf{点云构建策略优化:} 如何更有效地将时间序列转化为适用于TDA的点云表示,特别是时间延迟嵌入的参数优化(如嵌入维度和时间延迟的选择)、替代性嵌入技术的适用性等,仍是研究的重点。
    \item \textbf{特征提取与表示:} 从持续图中提取最具判别力且鲁棒的特征表示,如统计描述符、基于贝蒂数的表示,或更复杂的向量化方法(如持久性图像、持久性景观、持久性中心等),以及如何有效融合不同同调维度的信息,尚有许多问题值得深入探讨和比较分析~。
    \item \textbf{特定类型时间序列的处理:} 针对具有复杂时序依赖、强非线性、高噪声或需要融入领域知识的特定类型时间序列,其拓扑表示构建和特征提取方法仍有进一步优化的空间~。
    \item \textbf{理论与可解释性:} 深化对时间序列动力学行为与其拓扑特征之间精确关系的理论理解,为分类结果提供更强的统计保证和可解释性,是推动该领域发展的重要方向。
    \item \textbf{计算效率与可扩展性:} 持续同调的计算复杂度仍然是应用于大规模数据集的瓶颈,开发更高效的算法和并行处理技术至关重要。
\end{itemize}
这些挑战构成了本论文后续研究的主要动机和方向,旨在对基于持续同调的时间序列分类算法进行系统性的梳理、关键技术分析以及特定方法的深入探讨。

\subsection{本文的主要研究内容与目标}
针对上述研究背景、意义及现有研究的进展与挑战,本论文的核心研究内容聚焦于基于持续同调的时间序列分类算法的系统性梳理、关键技术分析以及特定方法的深入探讨。本论文首先将系统阐述相关的理论基础,详细介绍时间序列、点云、单纯形与单纯复形、过滤、同调群与贝蒂数、持续同调群与持续图等核心数学与拓扑学概念,并阐明从点云构建过滤复形(如Vietoris-Rips 复形和 Witness 复形)的基本原理与实例,为后续章节奠定坚实的理论基础~。

其次,本论文将深入综述与分析时间序列的拓扑表示构建方法。在这一部分,将重点介绍并分析两种具有代表性的时间序列拓扑表示构建策略:即基于持续同调的时间序列倾斜处理方法(PHTSI)~,探讨其如何通过时间倾斜增强时序信息的捕获;以及基于图的特征化时间序列表示方法,分析其如何灵活融入领域知识。同时,也将系统回顾经典的时间延迟嵌入(TDE)技术,深入讨论其理论基础(Takens嵌入定理)以及关键参数(嵌入维度d和时间延迟$\tau$)的选择策略与常用方法(如FNN、Cao方法、ACF、AMI等)~。此外,还会对其他替代性嵌入技术(包括值-差分嵌入、导数嵌入、PCA嵌入、多变量嵌入、非均匀嵌入)的原理、优缺点及适用场景进行概述~。

再次,本论文将全面梳理与比较基于持续同调的特征构造方法。此部分将系统介绍将持续同调的计算结果(主要是持续图)转化为适用于机器学习分类器的数值特征的各种主流方法,具体包括持久性图的统计描述符,基于贝蒂数的表示(如贝蒂曲线),以及多种基于函数或映射的向量化方法,如持久性熵(PE)、持久性图像(PI)、持久性景观(PL)和持久性中心(PC)等~。同时,对这些方法的原理、特性、计算复杂度及其在文献中的应用效果进行比较和讨论,并结合基于子窗口聚合的鲁棒PH特征提取策略进行分析~。

基于上述研究内容,本论文的研究目标主要体现在以下几个方面:
\begin{itemize}
    \item 为基于持续同调的时间序列分类提供一个相对全面和系统的理论与方法学综述,帮助读者理解该领域的核心概念、关键技术和研究进展~。
    \item 通过对不同时间序列拓扑表示构建策略(特别是点云构造方法)和拓扑特征提取技术的比较分析,揭示它们各自的优势、局限性以及在不同应用场景下的适用性,为相关研究者在方法选择上提供有益的参考~。
    \item 深入理解并阐明特定高级拓扑表示方法(如PHTSI、基于图的特征化方法)的设计思想和技术细节,评估其在特定类型时间序列分类任务中的潜力~。
\end{itemize}

本论文的后续章节安排如下:
\begin{itemize}
    \item \textbf{第二章:理论基础,}将系统介绍时间序列、点云以及持续同调等核心理论概念与计算方法,为后续研究奠定坚实的理论基础。
    \item \textbf{第三章:时间序列的拓扑表示构建方法,}将深入探讨并将原始时间序列数据转化为适用于拓扑数据分析的多种高维点云表示策略。
    \item \textbf{第四章:特征提取和分类,}将全面阐述如何从构建的拓扑表示中提取有效的持续同调特征及其他补充性特征,并将其应用于时间序列的分类任务。
    \item \textbf{第五章:总结和展望,}将对全文的研究工作进行凝练总结,并对该研究领域的未来发展方向进行展望。
\end{itemize}




