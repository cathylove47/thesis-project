\pagenumbering{arabic}
\section{绪论}
\subsection{研究背景与意义}
随着信息技术的飞速发展和数据采集能力的显著增强,时间序列数据已成为描述自然现象、社会经济活动和工程系统动态过程的核心载体。对这些时间序列数据进行有效分类,即时间序列分类(TSC),对于理解系统行为、预测未来趋势、实现智能决策具有至关重要的意义。正如文献中所广泛讨论的,时间序列分类在诸多领域展现出巨大的应用价值,例如在行为识别领域,利用智能设备的多模态传感器数据对日常活动(如步行、跑步、特定动作等)进行分类,有助于改善用户体验并实现即时响应\cite{JSJC202406009};在医疗健康领域,对心电图(ECG)信号、脑电图(EEG)信号或基因表达数据进行早期分类,能够辅助疾病(如心脏异常、哮喘、病毒感染、神经退行性疾病等)的诊断、预警和治疗方案制定\cite{1021736289.nh, mittal2017topological};在工业过程监控中,通过分析传感器产生的时间序列数据,可以早期发现设备故障(如泵泄漏、仪表失灵)、预测维护需求,从而降低运营风险和成本\cite{perea2015sliding,JSJZ202112028}。特别地,在许多时间敏感型应用中,如上述的疾病早期预警或工业故障即时发现,时间序列的早期分类能力显得尤为关键。

尽管现有时间序列分类方法众多,包括基于统计特征、距离度量(如动态时间规整DTW)、频域分析以及近年来兴起的深度学习模型等,但它们在面对具有复杂非线性动力学、多尺度结构特性或受噪声严重污染的时间序列时,仍可能面临特征提取不充分、模型解释性不足、对数据质量要求较高等挑战。例如,传统的统计特征可能难以捕捉数据中的非线性关系和全局结构,而深度学习方法虽然具有强大的特征学习能力,但在小样本或高维数据上可能面临过拟合的问题,且其“黑箱”特性往往限制了模型的可解释性。\cite{1021585745.nh}

在此背景下,拓扑数据分析(Topological Data Analysis, TDA)作为一种新兴的数学理论与数据分析工具,因其能够从复杂高维数据中提取稳健的、关于数据“形状”的全局和局部拓扑不变量,为时间序列分类研究提供了全新的视角和强大的分析手段。持续同调(Persistent Homology, PH)作为TDA的核心技术,通过量化拓扑特征在不同尺度下的演化,能够捕捉到传统方法难以揭示的时间序列内在结构模式。基于持续同调的方法通常具有较强的鲁棒性,对噪声和度量扰动不敏感,能够有效克服局部波动的干扰,并且其输出(如持续图或条形码)为理解数据的结构提供了可解释的途径。例如,在机械振动信号中,持续同调能够稳定识别出轴承早期故障所对应的环状拓扑模式;在医疗研究中,癫痫EEG信号中的短暂高频振荡现象也可能通过持续同调的H1维特征被精准捕获。因此,深入研究并发展基于持续同调的时间序列分类算法,对于提升复杂时间序列分析的性能、增强模型的可解释性,具有重要的理论价值和广阔的应用前景。

\subsection{拓扑数据分析简介及其应用于时间序列的动机}
拓扑数据分析(TDA)是一门利用拓扑学概念和工具来分析和理解复杂数据集内在结构与形状的新兴交叉学科。其核心思想在于,许多高维复杂数据尽管在几何上形态各异,但其潜在的拓扑结构(如连通性、环状结构、空腔等,对应于不同维度的“孔洞”)可能更为本质和稳定。持续同调是TDA中最核心和应用最广泛的工具之一 \cite{zomorodian2004computing, dey2022computational}。它通过构建数据的多尺度拓扑表示(通常是从点云数据出发,通过Vietoris-Rips复形、Witness复形或Čech复形等方法构造的过滤单纯复形序列),并追踪拓扑特征(如同调群的秩,即贝蒂数)在过滤过程中的“出生”(birth)与“死亡”(death)尺度,从而量化这些特征的“持续性”(persistence)。持续性长的特征通常被认为是数据中稳定且重要的结构,而持续性短的特征则可能对应噪声或细微结构。持续同调的计算结果通常以持续图(Persistence Diagram, PD)或等价的持续条形码(Persistence Barcode)的形式呈现,它们直观地总结了数据在不同尺度下的多尺度拓扑信息。

将TDA应用于时间序列分析的动机源于时间序列数据,特别是那些来源于复杂动力系统或具有内在周期性、递归性的序列,其本身往往蕴含着丰富的几何和拓扑结构。例如,一个周期性或准周期性时间序列经过适当的嵌入(如时间延迟嵌入,详见第三章)后,可以在高维空间中形成具有特定拓扑特征(如环状)的点云吸引子\cite{takens2006detecting, perea2015sliding}。与传统统计特征相比,TDA能够捕捉到超越简单均值、方差的全局形状信息;与某些依赖特定模式匹配的方法(如基于形状的方法)相比,拓扑特征对变形和噪声更不敏感;与一些黑箱模型(如部分深度学习模型)相比,持续图等TDA输出提供了更具可解释性的结构洞察。因此,利用持续同调提取时间序列的拓扑特征,有望捕捉到那些对分类任务至关重要的、稳健的、且具有内在结构意义的信息,从而为时间序列分类提供一种全新的、有效的特征工程途径。

近年来,国内外学者已开始积极探索将TDA应用于时间序列分析与分类,并在多个领域(如生理信号分析\cite{3}、金融时间序列、动态系统识别\cite{mittal2017topological}等)展现出其潜力。这些研究主要集中在如何有效地将时间序列转化为适用于TDA的表示(如点云构造,详见第三章),以及如何从持续同调的输出(如持续图)中提取有效的特征用于分类任务(详见第四章)。例如,Karan和Kaygun \cite{3} 利用持续同调从生理信号中提取拓扑特征进行应激识别。Perea和Harer \cite{perea2015sliding} 则通过滑动窗口和持续同调分析信号的周期性。海彤 \cite{1021736289.nh} 在其学位论文中也探讨了基于拓扑分析的时间序列分类方法。

然而,尽管已有诸多尝试,关于点云构造策略的选择(包括时间延迟嵌入的参数优化、替代嵌入技术的适用性等)、以及从持续图中提取最具判别力且鲁棒的特征表示(如统计描述符、基于贝蒂数的表示、或更复杂的向量化方法如持久性图像、持久性景观等)等方面,仍有许多问题值得深入探讨和比较分析。特别是针对特定类型时间序列(如具有复杂时序依赖或需要融入领域知识的序列)的拓扑表示构建和特征提取方法,尚有进一步优化的空间。这构成了本论文后续研究的主要动机和方向。

\subsection{研究内容与目标}
针对上述研究背景、意义及现有研究的进展与挑战,本论文的核心研究内容聚焦于基于持续同调的时间序列分类算法的系统性梳理、关键技术分析以及特定方法的深入探讨。
本论文首先将系统阐述相关的理论基础,详细介绍时间序列、点云、单纯形与单纯复形、过滤、同调群与贝蒂数、持续同调群与持续图等核心数学与拓扑学概念,并阐明从点云构建过滤复形(如Vietoris-Rips复形和Witness复形)的基本原理与实例,为后续章节奠定坚实的理论基础。

其次,本论文将深入综述与分析时间序列的拓扑表示构建方法。在这一部分,将重点介绍并分析两种具有代表性的时间序列拓扑表示构建策略:即基于持续同调的时间序列倾斜处理方法(PHTSI)\cite{JSJC202406009},探讨其如何通过时间倾斜增强时序信息的捕获;以及基于图的特征化时间序列表示方法\cite{2},分析其如何灵活融入领域知识。同时,也将系统回顾经典的时间延迟嵌入(TDE)技术,深入讨论其理论基础(Takens嵌入定理)以及关键参数(嵌入维度$d$和时间延迟$\tau$)的选择策略与常用方法(如FNN、Cao方法、ACF、AMI等)。此外,还会对其他替代性嵌入技术(包括值-差分嵌入、导数嵌入、PCA嵌入、多变量嵌入、非均匀嵌入)的原理、优缺点及适用场景进行概述。

再次,本论文将全面梳理与比较基于持续同调的特征构造方法。此部分将系统介绍将持续同调的计算结果(主要是持续图)转化为适用于机器学习分类器的数值特征的各种主流方法,具体包括持久性图的统计描述符,基于贝蒂数的表示(如贝蒂曲线),以及多种基于函数或映射的向量化方法,如持久性熵(PE)、持久性图像(PI)、持久性景观(PL)和持久性中心(PC)等。同时,对这些方法的原理、特性、计算复杂度及其在文献中的应用效果进行比较和讨论,并结合基于子窗口聚合的鲁棒PH特征提取策略\cite{3}进行分析。


基于上述研究内容,本论文的研究目标主要体现在以下几个方面。其一,为基于持续同调的时间序列分类提供一个相对全面和系统的理论与方法学综述,帮助读者理解该领域的核心概念、关键技术和研究进展。其二,通过对不同时间序列拓扑表示构建策略(特别是点云构造方法)和拓扑特征提取技术的比较分析,揭示它们各自的优势、局限性以及在不同应用场景下的适用性,为相关研究者在方法选择上提供有益的参考。其三,(如果适用)深入理解并阐明特定高级拓扑表示方法(如PHTSI、基于图的特征化方法)的设计思想和技术细节,评估其在特定类型时间序列分类任务中的潜力。其四,(如果适用)通过实验验证所研究方法的实际性能,并对其结果进行分析讨论,为未来进一步优化基于TDA的时间序列分类算法提供一些有益的启示和方向。

本论文的后续章节安排如下:第二章详细介绍本研究相关的数学理论基础。第三章深入探讨将时间序列数据转化为适用于拓扑数据分析的各种表示方法。第四章全面综述从持续同调结果中提取特征的技术,并简要介绍其他类型的常用时间序列特征。第五章(如果适用)将详细阐述本文重点研究的PHTSI算法框架与实现细节。第六章(如果适用)将呈现相关的实验设计、结果分析与讨论。最后,第七章对全文进行总结,并展望未来的研究方向。



