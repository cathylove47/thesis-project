\section{TDA特征在时间序列分类模型中的应用与集成}
\subsection{将TDA特征输入传统机器学习模型}
介绍如何将量化的TDA特征(如统计量,PL/PI的向量化表示)输入KNN,SVM,随机森林等模型
讨论距离度量选择的重要性:例如在特征空间中使用欧氏距离,结合其他时间序列距离如DTW

\subsection{将TDA特征输入深度学习模型}
重点介绍将PI作为图像输入CNN进行分类的方法
简要提及将贝蒂序列,PL曲线或其他序列型拓扑特征输入1D CNN或 RNN模型

\subsection{结合TDA特征与原始时间序列或其他表示}
解释为何TDA特征可能不足以完全描述时间序列
TDA特征侧重形状,但可能忽略幅值,频率等重要信息 
结合是为了提供更全面,互补的表示

常见的结合方法: 特征拼接:简单的将TDA特征向量与其他特征向量拼接后输入分类器

集成度量: 在距离层面对不同特征的距离进行加权或组合

多通道模型: 设计深度学习网络,在不同通道处理不同类型的输入特征,并在网络的后期融合

