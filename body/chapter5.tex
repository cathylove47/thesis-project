\section{总结和展望}

\subsection{研究总结}
本论文系统地探讨了基于持续同调的时间序列分类算法,旨在深入理解如何从时间序列数据中提取鲁棒且具有解释性的拓扑特征,并应用于分类任务。研究工作全面梳理了相关的理论基础,详细综述了时间序列拓扑表示的构建方法,并对从持续同调中提取特征的关键技术进行了比较与分析。

在理论基础层面(第二章),本研究首先明确了时间序列和点云的基本定义,为后续的拓扑分析奠定了基础。随后,系统阐述了持续同调的核心概念,包括单纯形与单纯复形的构造原理,通过过滤(如Vietoris-Rips复形和Witness复形)从点云数据生成多尺度拓扑表示的过程,以及链群、边缘算子、同调群和贝蒂数的代数定义。核心在于引入了持续同调群与持续图,它们能够有效地捕捉拓扑特征在不同尺度下的出生、死亡及持续性信息,是后续特征提取与分析的关键工具。同时,本章也简要提及了同伦这一重要的拓扑概念,以期为读者提供更广阔的拓扑学视角。

在时间序列拓扑表示构建方法层面(第三章),本研究首先突出了两种具有代表性的策略:其一是基于持续同调的时间序列倾斜处理方法(PHTSI)\cite{JSJC202406009},该方法通过对子序列施加时间相关的加权变换,旨在增强对时序依赖信息的捕获;其二是基于图的特征化时间序列表示方法\cite{2},它允许灵活地融入领域知识以定制化持久同调的分析过程。为支撑这些方法的理解,本章详细回顾了经典的时间延迟嵌入(TDE)技术,包括其理论基石Takens嵌入定理\cite{takens2006detecting},并深入探讨了嵌入维度$d$和时间延迟$\tau$这两个关键参数的选择策略与常用方法(如FNN\cite{rhodes1997false}、Cao方法\cite{cao1997practical}、ACF\cite{kantz2003nonlinear}、AMI\cite{wallot2018calculation}等)。此外,本章还对其他替代性嵌入技术,如值-差分嵌入、导数嵌入\cite{lekscha2018phase, lainscsek2015delay}、主成分分析(PCA)嵌入\cite{broomhead1986extracting, gibson1992analytic}、多变量嵌入\cite{barnard2001embedding}以及非均匀嵌入\cite{jia2019detecting, jia2020refined}等进行了概述,分析了它们各自的原理、优势与局限性。这些方法共同构成了从原始时间序列到TDA可处理的点云或图结构表示的多样化途径。

在拓扑特征提取与分类层面(第四章),本研究全面梳理了从持续同调的计算结果(主要是持续图)中构造数值特征的各种主流技术。具体讨论了持久性图的统计描述符\cite{mittal2017topological},基于贝蒂数的表示(如贝蒂曲线),以及多种基于函数或映射的向量化方法,包括持久性熵(PE)\cite{atienza2020stability, atienza2018stability}、持久性图像(PI)\cite{adams2017persistence}、持久性景观(PL)\cite{1}和持久性中心(PC)\cite{JSJC202406009}等。对这些方法的原理、稳定性、计算复杂度及其在文献中的应用效果进行了比较,并结合了基于子窗口聚合的鲁棒PH特征提取策略\cite{3}进行分析。此外,本章还简要介绍了时间序列的其他常用特征提取方法(如时域统计特征、频域特征、时域模式和形状特征以及基于模型的特征),并概述了在获得特征向量后常用的机器学习分类器。

(若您的论文中PHTSI算法是核心贡献,且实验主要围绕它展开,可以在此补充一句总结其表现,例如:)
特别地,通过对PHTSI算法的深入分析(或实验验证),揭示了其在特定类型时间序列数据上捕捉时序动态和拓扑结构方面的潜力。

总体而言,本论文为基于持续同调的时间序列分类提供了一个较为系统和全面的视角,从理论基础、表示构建、特征提取到分类应用进行了梳理和探讨,旨在为相关领域的研究者提供有益的参考和借鉴。

\subsection{展望}
尽管基于持续同调的时间序列分类方法已展现出其独特的优势和广阔的应用前景,但该领域仍存在许多值得进一步探索和研究的方向。

首先,在时间序列的拓扑表示构建方面,仍有提升空间。虽然时间延迟嵌入是经典且常用的方法,但其参数选择的敏感性以及对复杂多尺度动态捕捉的局限性依然是挑战。未来研究可继续探索更自适应、更鲁棒的嵌入技术,例如结合机器学习方法自动优化嵌入参数,或者发展能够更有效处理非平稳、高噪声时间序列的特定嵌入策略。对于PHTSI和基于图的特征化等新颖方法,其理论性质、参数影响以及在更广泛数据集上的适用性仍需进一步深入研究和验证。

其次,从持续图中提取更具判别力且计算高效的特征表示仍然是核心问题。现有的向量化方法各有优劣,未来可以探索结合多种表示的优点,或者利用深度学习技术直接从持续图(或其他拓扑描述符)中学习特征表示,例如发展针对持续图的图神经网络或卷积神经网络结构。同时,如何将时间序列的动态演化信息更紧密地融入到拓扑特征的构建中,以实现时序信息与拓扑结构信息的深度融合,也是一个重要的研究方向。

再次,关于算法的计算效率和可扩展性。持续同调的计算本身,特别是对于大规模点云或高维复形,仍然是计算瓶颈。虽然已有如Witness复形等简化方法,但探索更快速的过滤构建算法、近似持续同调计算方法以及适用于大规模时间序列数据集的并行化TDA流程,对于推动该技术在实际应用中的普及至关重要。

此外,理论层面的研究也亟待深化。例如,如何更深刻地理解特定时间序列动力学行为(如混沌、分岔)与其拓扑特征(如特定维度持续同调的生灭模式)之间的精确数学关系;如何从理论上指导针对特定类型时间序列(如金融、生理信号)的拓扑表示和特征选择;以及如何为基于TDA的分类结果提供更强的统计显著性保证和可解释性,都是未来值得关注的理论问题。

最后,在应用拓展方面,基于持续同调的时间序列分类方法有望在更多领域发挥作用。除了已有的医疗诊断、工业监控等,还可以探索其在气候变化分析、社会网络动态演化、自然语言处理(如文本序列的情感分类)等新兴领域的应用。结合多模态数据,将TDA与其他数据分析技术(如深度学习、强化学习)相融合,构建更强大的混合模型,也可能成为未来的一个重要趋势。

综上所述,基于持续同调的时间序列分类是一个充满活力且快速发展的研究领域。通过在理论、方法、计算和应用等多个层面持续创新,有望进一步释放拓扑数据分析在理解和处理复杂时间序列数据方面的巨大潜力。

