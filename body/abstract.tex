\section*{\zihao{-2} \centering 摘 ~~ 要}
% 生成序号列表
 \begin{enumerate}
 \item 简述时间序列分类的重要性以及传统方法的局限性
 \item 引入拓扑数据分析作为一种新兴方法,在捕捉数据内在结构方面的独特潜力
 \item 明确核心:深入综述和比较分析点云构造策略和从PD中提取有效特征的技术
 \item 概述综述的主要内容,对不同方法的原理,优缺点,参数影响和适用性进行比较
 
 \end{enumerate}





\vskip0.5cm

{\zihao{4} \heiti 关键词: } \zihao{-4}拓扑数据分析,时间序列分类,持续同调,点云构造,时间延迟嵌入,特征提取,持久性图
% \addcontentsline{toc}{section}{摘要}

\clearpage
\section*{\zihao{-2} \centering \textbf{Abstract} }
   %用了Times New Roman字体来美化观感
   
here is English Abstract

\vskip0.5cm

\textbf{\zihao{4} Key Words:} Multi-Agent Deep Reinforcement Learning, Formation Control, Path Planning
% \addcontentsline{toc}{section}{Abstract}




