\section{时间序列的拓扑表示构建方法}
\subsection{时间序列转化为TDA可处理的表示}
时间序列分析在金融、医疗、工业等领域具有广泛应用,但其非线性、混沌和周期性特性使得传统方法(如统计分析、傅里叶变换)难以捕捉高维拓扑结构或动态特性。例如,脑电图(EEG)信号的复杂时序模式需要更强大的分析工具来提取隐含信息。拓扑数据分析(TDA)通过持久同调(Persistent Homology)提取数据的拓扑特征,如连接组件和孔洞,对噪声鲁棒且能揭示复杂数据的内在模式,特别适合处理非线性时间序列。然而,TDA要求数据以点云或复形形式输入,而时间序列为一维数值序列,无法直接应用。因此,需通过时间延迟嵌入(TDE)等方法将时间序列转化为高维点云,以捕捉动态系统的吸引子或拓扑特性。本章将系统介绍时间序列到TDA表示的转化方法,包括时间延迟嵌入(TDE)、值-差分散入等,分析其数学原理、实现步骤和适用场景,为后续特征提取和分类任务奠定基础。本章结构如下:首先介绍TDE的理论基础和参数选择方法,其次探讨值-差分散入等替代方法,最后比较不同方法的优缺点及适用性。
\subsection{状态空间重构}
我们需要将观测到的时间序列数据重构系统动力学状态空间,并构建其拓扑表示的方法.在许多科学和工程领域,我们面对的是复杂的动态系统,但往往只能观测到系统的一个或少数几个变量随着时间而变化,这种观测导致我们难以理解分析系统完整行为,所以我们需要从有限的,可能带有噪声的观测数据中,推断出驱动系统演化的潜在高维信息.传统的线性时间序列分析方法虽然在某些情况下有效,但往往难以捕捉复杂系统所固有的非线性特征和相互作用.状态空间重构是一个非常的好的解决办法,其基本洗漱是将低维的时间序列数据,通过特定的嵌入技术,转化为一个高维空间中的几何对象,这个重构出的状态空间在理想情况下能保持原始系统动力学的主要拓扑和几何特性.在众多状态空间重构技术中,时间延迟嵌入(TDE)扮演着基石性的角色.
\subsubsection{Takens 嵌入定理}
TAKENs嵌入定理(Takens' Embedding Theorem)是由荷兰数学家Floris Takens在1981年提出的一个重要数学定理,主要应用于动力系统和时间序列分析领域。它提供了一种从单一标量时间序列数据中重构原始动力系统相空间的方法,尤其在研究混沌系统和非线性动力学时具有深远意义。

在介绍定理之前我们先来介绍两个基本概念:
\begin{itemize}
    \item \textbf{相空间(Phase Space)}: 动力系统的状态空间,描述系统所有可能状态的集合。对于一个n维动力系统,相空间是n维的。
    \item \textbf{吸引子(Attractor)}: 动力系统在长时间演化后趋向的状态或轨迹集合。吸引子可以是点、周期轨道或更复杂的结构,如奇异吸引子。
\end{itemize}
TAKENS嵌入定理的核心思想是,通过对时间序列进行适当的嵌入,可以在高维空间中重构出原始动力系统的相空间结构。具体来说,假设我们有一个一维时间序列$x(t)$,它是一个动力系统在时间上的投影。TAKENS定理指出,如果我们选择合适的嵌入维度$m$和时间延迟$\tau$,那么通过以下方式构造的$m$维向量序列可以近似重构原始动力系统的相空间:
\begin{equation}
    \mathbf{x}_i = (x(t_i), x(t_{i+\tau}), x(t_{i+2\tau}), \ldots, x(t_{i+(m-1)\tau}))
\end{equation}
其中,$i$是时间序列的索引,$t_i$是时间点,$m$是嵌入维度,$\tau$是时间延迟。

TAKENS定理的直观解释是:由于确定性动力系统中各个状态变量是相互耦合的,当前观测值及其过去(或未来)的延迟值序列中,蕴含了关于当前未被直接观测到的其他状态变量的信息.
补充: sauer等人将嵌入维度$m$的下限给出为$m\geq 2d+1$,其中$d$是原始系统的维数.这个定理的一个重要结论是,只要嵌入维度足够大,就可以通过重构的相空间来恢复原始系统的动力学行为,包括吸引子结构和混沌特性.

尽管TAKENS定理在理论上极其重要,但在实际应用中由于真实世界的时间序列通常是有限长度的,离散采样的,并且不可避免的受到噪声污染,这些因素导致定理的严格条件往往无法满足.此外,系统的平稳性也是一个重要的假设,非平稳数据会给重构带来困难.因此TAKENS定理保证了重构的可能性,但是在实践中,我们需要依赖经验方法来选择合适的嵌入参数,如嵌入维度$m$和时间延迟$\tau$.,以期获得一个对原始动力学有用的近似重构,即使它不完全满足微分同胚的条件.参数的选择变成了一个关键的实践问题,直接影响重构质量.



\subsubsection{时间延迟嵌入 (TDE)}
时间延迟嵌入(Time Delay Embedding, TDE) 是一种将一维时间序列数据转化为高维点云的技术,基于 Takens 嵌入定理。TDE 的基本思想是通过引入时间延迟和嵌入维度,将时间序列中的每个观测值与其过去的观测值组合成一个高维向量,从而重构出系统的相空间。这种方法能够捕捉到系统的非线性动力学特征,如周期性、混沌行为等。
TDE 的基本原理是将一维时间序列 \( s(t) \) 通过引入时间延迟 \( \tau \) 和嵌入维度 \( d \),转化为高维向量:
\begin{equation}
\vec{x}(t) = \left[ s(t), s(t - \tau), s(t - 2\tau), \dots, s(t - (d - 1)\tau) \right]
\end{equation}
这些向量构成的集合形成了一个高维点云,即重构的相空间。Takens 定理指出,当嵌入维度 \( d \geq 2m + 1 \)(其中 \( m \) 为原始系统的维数)时,重构的相空间与原始系统的相空间拓扑等价。这种重构使得我们可以从单一观测变量中恢复系统的非线性动力学行为,例如吸引子结构或混沌特性。点云中的每个点代表系统在某时刻的状态,而点云的几何形状则反映了系统的长期演化规律。


\subsubsection{嵌入维度参数\(d\)的选择}
嵌入维度\(d\)决定了重构相空间的维度,直接影响了点云的展开程度.选择适当的d可以捕捉到系统的非线性特征,但过高的m会导致维数灾难,使得数据稀疏,难以提取有效的拓扑特征.常用的方法包括:

\begin{table}[h!] % 使用 [h!] 尝试将表格放置在此处
    \centering % 表格居中
    \caption{时间延迟嵌入嵌入维度($m$)选择方法对比(不含优点和缺点)}
    \label{tab:embedding_dimension_methods_no_pros_cons}
    \begin{tabular}{
      >{\raggedright\arraybackslash}m{3cm} % 方法名称 (左对齐,自动换行)
      >{\raggedright\arraybackslash}m{4.5cm} % 核心原理 (左对齐,自动换行)
      >{\raggedright\arraybackslash}m{5cm}  % 原理阐释 (左对齐,自动换行)
      >{\raggedright\arraybackslash}m{4cm}  % 常用判据 (左对齐,自动换行)
    }
    \toprule % 顶部线条
    \textbf{方法名称} & \textbf{核心原理}  & \textbf{原理阐释}  & \textbf{常用判据/解释} \\
    \midrule % 中间线条
    
    伪近邻法 (FNN) & 邻近点在维度增加时的相对距离变化  & 维度不足时,投影导致假邻居;维度足够时,假邻居会散开,真邻居保持靠近  & \%FNN随 $m$ 增加首次降至零或小阈值  \\
    \addlinespace % 增加行间距
    
    Cao方法 & 近邻点对距离相对变化率的平均值 ($E1(m)$)  & 旨在改进FNN,减少对阈值的依赖;观察变化率何时稳定  & 绘制 $E1(m)$ 曲线,寻找其趋于平稳的 $m$ 值 。$E2(m)$ 用于区分确定性与随机性  \\
    \addlinespace
    
    符号动力学/熵方法 & 符号序列的熵  & 利用熵度量信息量或依赖性随参数的变化;将 $m$ ($p$), $τ$ 与最优时间窗口 $\tau_w$ 关联  & 寻找使熵最大化($\tau^*$)和最小化($\tau_w$)的延迟,通过 $\tau_w=(p-1)\tau^*$ 定 $p$  \\
    \addlinespace
    
    Takens/Sauer 定理 (理论基础) & 保证嵌入的微分同胚性  & 足够高的维度能“展开”吸引子,保持拓扑结构  & 理论要求 $m > 2d$ (Takens) 或 $m > 2d_A$ (Sauer),其中 $d$ 为系统维数,$d_A$ 为吸引子盒维数 \\
    
    \bottomrule % 底部线条
    \end{tabular}
    \par % 确保表格后的文本在新行开始
    \vspace{0.5cm} % 在表格后增加一些垂直空间
    \textit{注意:实际应用中选择嵌入维度 $m$ 通常是一个启发式过程,可能需要结合数据特性和析目标进行权衡 。FNN 和 Cao 方法旨在估计最小且足够的 $m$ 。}
    \end{table}
    
\subsubsection{时间延迟参数$\tau$的选择}


在利用时间延迟嵌入技术重构复杂系统相空间的过程中,除了选择合适的嵌入维度 $m$ 之外,时间延迟参数 $\tau$ 的选取同样是\textbf{至关重要}的一环。它直接决定了构成嵌入向量 $\vec{y}(t) = [x(t), x(t+\tau), \dots, x(t+(m-1)\tau)]$ 各分量之间的时间间隔。一个恰当的 $\tau$ 值是保证重构相空间能够有效“展开”原始动力系统吸引子,并保持其拓扑结构不变性的关键因素之一。

选择一个过小的 $\tau$ 值,会导致嵌入向量的相邻分量之间线性相关性过强,信息冗余度高。此时,$x(t)$ 与 $x(t+\tau)$ 非常接近,它们并未提供足够多的“新”信息来区分邻近的轨迹段。结果是,重构出的吸引子可能仍然是“折叠”或“挤压”在一起的,未能充分展现其在高维空间中的真实几何形态,这无疑会扭曲我们对系统动力学行为的理解。

相反,如果选择一个过大的 $\tau$ 值,虽然可以保证各分量之间的线性无关性甚至统计独立性,但 $x(t)$ 与 $x(t+(m-1)\tau)$ 之间可能已经失去了系统内在的动力学关联。这相当于在时间上跳跃得太远,使得嵌入向量无法捕捉到系统轨迹的连续演化特征和确定性结构,重构出的相空间可能变得杂乱无章,甚至接近随机噪声的表现,同样无法真实反映原始动力学。

因此,$\tau$ 的选择面临着一个\textbf{核心的权衡}:既要足够大以确保嵌入向量的各分量包含足够独立的信息,从而有效展开吸引子;又要足够小以保留系统随时间演化的内在关联性与动力学规律。一个不恰当的 $\tau$ 值将直接影响后续所有基于重构相空间的分析,包括分形维数的计算、Lyapunov 指数的估计、系统的预测精度以及噪声抑制的效果等。鉴于其对整个相空间重构质量和后续分析有效性的\textbf{决定性影响},如何合理地选择时间延迟 $\tau$ 成为了时间序列非线性分析中的一个基本且关键的问题。下面将展示几种常用的 $\tau$ 值选择策略及其背后的原理。


\begin{table}[h!] % 使用 [h!] 尝试将表格放置在此处
    \centering % 表格居中
    \caption{时间延迟嵌入时间延迟 ($\tau$) 选择方法对比} % 表格标题
    \label{tab:time_delay_tau_methods_compare} % 用于交叉引用的标签
    \begin{tabular}{
      >{\raggedright\arraybackslash}m{3cm}    % 方法名称 (左对齐,自动换行,宽度 3cm)
      >{\raggedright\arraybackslash}m{4.5cm}  % 核心原理 (左对齐,自动换行,宽度 4.5cm)
      >{\raggedright\arraybackslash}m{5cm}   % 原理阐释 (左对齐,自动换行,宽度 5cm)
      >{\raggedright\arraybackslash}m{4cm}    % 常用判据 (左对齐,自动换行,宽度 4cm)
      % 注意:总宽度 3+4.5+5+4 = 16.5cm,加上列间距可能超出典型 A4 页边距下的文本宽度
      % 如遇表格超出页面,请尝试减小 m{} 列的宽度值
    }
    \toprule % 顶部线条 (来自 booktabs)
    \textbf{方法名称} & \textbf{核心原理} (分析的度量) & \textbf{原理阐释} (为何有效) & \textbf{常用判据/解释} \\
    \midrule % 中间线条 (来自 booktabs)
    
    自相关函数 (ACF) & 时间序列与其滞后版本间的线性相关性  & 寻找 $x(t)$ 与 $x(t+\tau)$ 线性无关的最小延迟  & $\tau$ 取 ACF 首次降至零  或首次降至 $1/e \approx 0.37$  的值 \\
    \addlinespace % 增加行间距 (来自 booktabs)
    
    平均互信息 (AMI) & $x(t)$ 与 $x(t+\tau)$ 之间的统计依赖性 (信息论度量)  & 寻找 $x(t)$ 与 $x(t+\tau)$ 共享信息量最少的延迟,以捕捉非线性结构  & $\tau$ 取 AMI 函数 $I(\tau)$ 的第一个局部最小值  \\
    \addlinespace
    
    C-C 方法 (基于关联积分) & 分析关联积分在不同 $\tau, m$ 下随邻域半径 $r$ 的变化 [...] & 通过检验不同时间窗内数据的统计独立性与几何展开程度寻找最优参数  & 寻找特定统计量(如 $\Delta \bar{S}_2(t)$ 或 $\sigma_{cor}(t)$)首次穿零或达极小值对应的 $t$ (即 $\tau$)  \\
    \addlinespace
    
    几何/直观方法 & 观察时间序列图或相图的特征时间尺度  & 基于对系统动力学行为的先验知识或直观观察估计 & $\tau$ 常取为主要周期或特征时间的某个分数(如 1/4 到 1/10)\\
    
    \bottomrule % 底部线条 (来自 booktabs)
    \end{tabular}
    \par % 确保表格后的文本在新行开始
    \vspace{0.5cm} % 在表格后增加一些垂直空间
    \textit{注意:选择 $\tau$ 与选择 $m$ 密切相关。实际应用中 $\tau$ 的选择常带有启发性,需测试不同值的影响 [...]。AMI 通常被认为是较优选择 [...]。[...]处请替换为实际引用。}
    \end{table}

\textbf{总结}: 通过前面的讨论可知,为了从一维时间序列 $x(t)$ 中成功重构出能够反映原始系统动力学特性的相空间,合理地选择嵌入维度 $m$ 和时间延迟 $\tau$ 是两个相辅相成且至关重要的步骤。这两个参数共同决定了重构相空间的几何形态与质量。

嵌入维度 $m$ 的选择旨在确保重构空间具有足够的维度,以完全“展开”动力系统的吸引子,避免因维度不足导致的轨迹交叉和结构误判(即伪近邻现象)。我们探讨了多种估计最小充分嵌入维度的方法,包括基于几何直观的伪近邻法 (FNN) 及其改进(如 Cao 方法),以及基于系统理论的 Takens/Sauer 定理等。FNN 及其变种通过考察邻近点在维度增加时的相对距离变化来寻找合适的 $m$,而理论定理则提供了嵌入有效性的数学保证,但其给出的往往是维度的上限而非最优实用值。

时间延迟 $\tau$ 的选择则关注于如何设置嵌入向量中各分量之间的时间间隔,以在保证分量间足够统计独立性(从而有效利用 $m$ 维空间)与保留系统时间演化内在关联性之间取得平衡。$\tau$ 过小导致信息冗余,$\tau$ 过大则可能丢失动力学信息。常用的 $\tau$ 选择方法包括分析时间序列自相关性(如自相关函数 ACF 法,主要关注线性相关)和信息论依赖性(如平均互信息 AMI 法,能更好地处理非线性相关),以及一些试图结合几何特性或同时考虑 $m$ 和 $\tau$ 的方法(如 C-C 方法)。其中,AMI 方法因其对非线性动力学的适应性而被广泛推荐。

值得强调的是,$m$ 和 $\tau$ 的选择并非完全独立的过程。它们共同定义了重构所利用的时间跨度,即所谓的“时间窗口” $\tau_w = (m-1)\tau$ [...]。某些方法(如 C-C 方法)正是着眼于这个时间窗口的优化。更重要的是,实际应用中参数的选择往往带有启发性,不存在一个适用于所有类型时间序列(如长/短、含噪/纯净、周期/混沌)的通用“万能”方法。

因此,在实践中,研究者通常不会仅仅依赖单一方法的建议值,而是倾向于尝试应用多种不同的方法(例如结合几何类、信息论类等手段)来估计 $m$ 和 $\tau$,并对结果进行相互比较和验证。同时,必须充分考虑所分析时间序列的具体特性,例如其采样频率、噪声水平以及内在的特征时间尺度等,这些因素都可能影响不同选择方法的表现和适用性。进行参数敏感性分析也极为关键:考察在一个根据初步方法确定的合理 $m$ 和 $\tau$ 参数值邻域内,后续计算得到的动力学不变量(如关联维、Lyapunov指数)或模型的预测误差是否保持稳定,是验证参数选择鲁棒性的重要手段。最终的选择决策,应当是在综合考量多种方法建议、数据自身特点、参数敏感性测试结果以及具体科学研究目标的基础上作出的。

总而言之,只有通过这样细致、多角度的比较、选择和验证过程,才能更有信心地确保所构建的重构相空间是原始动力系统的一个有效且可靠的拓扑等价表示,从而为后续深入的非线性动力学分析和应用奠定坚实的基础。[...]


% 注意:在你的论文中,你可能需要为这些方法添加参考文献,
% 例如引用 Fraser & Swinney (1986) 关于互信息法的论文,
% 或其他关于吸引子重构和参数选择的综述性文献或教科书。

\subsection{值-差分嵌入}
值-差分嵌入(Value-Difference Embedding)是一种将时间序列转化为高维点云的替代方法。与传统的时间延迟嵌入方法不同,值-差分嵌入不仅考虑了时间序列的当前值,还考虑了相邻时间点之间的差异。这种方法可以更好地捕捉时间序列中的动态变化和局部结构。

\subsubsection{方法原理}
在这篇论文 \cite{Yan2024PHTSI} 中,值差分嵌入特指将一个单变量时间序列 \( X = (x_1, x_2, \dots, x_n) \) 转换为一个二维点集(点云)。转换方式是生成一系列点 \( (x_i, x_{i+1} - x_i) \),其中 \( i \) 从 1 变化到 \( n-1 \)。得到的点云 \( P \) 就是 \( \{ (x_i, x_{i+1} - x_i) \mid i = 1, \dots, n-1 \} \) [cite: 398, 418]。

\subsubsection{动机}
提出这种特定二维嵌入方法的主要动机包括:
\begin{itemize}
    \item \textbf{捕捉动态变化}: 该方法旨在通过同时包含时间序列的当前值 ($x_i$) 和其一阶差分 ($x_{i+1} - x_i$),来更好地展现时间序列在周期内和周期之间的动态变化特征 [cite: 398, 415]。差分项直接反映了序列的局部变化趋势。
    \item \textbf{简化嵌入过程}: 相比于传统的时间延迟嵌入 (TDE) 需要仔细选择嵌入维度 $d$ 和时间延迟 $\tau$ (这些参数的选择对重构效果影响很大,且在不同数据集上标准不一,计算复杂 [cite: 415]),这种值-差分嵌入直接将一维时间序列映射到固定的二维空间,避免了参数选择的困难。
    \item \textbf{为拓扑分析提供基础}: 构建的二维点云可以直接作为拓扑数据分析(如计算持续同调)的输入,以提取时间序列数据中蕴含的拓扑结构特征 [cite: 398, 415]。
\end{itemize}

\subsubsection{优点与缺点}
基于其原理和动机,可以总结该方法的优缺点如下:

\textbf{优点:}
\begin{itemize}
    \item \textbf{实现简单}: 将一维时间序列映射到二维点云的过程直观且易于实现,不需要复杂的参数调优 [cite: 415]。
    \item \textbf{捕捉局部动态}: 通过引入差分项,能够直接反映时间序列的增减变化和速率信息,有助于捕捉局部动态特征 [cite: 398]。
    \item \textbf{固定低维输出}: 始终生成二维点云,可能使得后续的持续同调计算(尤其是在点云规模不大时)相对高效 [cite: 398]。
\end{itemize}

\textbf{缺点:}
\begin{itemize}
    \item \textbf{维度限制}: 将数据固定嵌入到二维空间,可能无法完全“展开”高维动态系统真实的吸引子结构。根据Takens定理,对于复杂系统,可能需要更高的嵌入维度才能保证拓扑等价性。这种二维嵌入可能丢失部分高维拓扑信息。
    \item \textbf{对噪声敏感性 (可能)}: 差分操作 ($x_{i+1} - x_i$) 可能会放大原始时间序列中存在的噪声,从而影响点云的结构和后续拓扑特征的稳定性。 (注:这一点是基于差分操作特性的推断,原文未明确强调)
    \item \textbf{信息整合有限}: 只考虑了当前值和紧邻的下一个值之间的关系,可能不如TDE(可以设置更大的 $\tau$)那样灵活地捕捉不同时间尺度上的依赖关系。
\end{itemize}

\subsection{基于时间序列函数或其变换的滤过}
\label{sec:func_filtration}
除了通过相空间重构构建点云,拓扑数据分析(TDA)亦可直接应用于时间序列函数本身或其变换。此类方法通过在函数的值域上定义滤过(filtration),进而利用持久性同调(persistent homology, PH)来捕捉与函数值演化相关的拓扑特征。
\subsubsection{方法原理}
\label{sec:func_filtration_principle}

令 $s(t)$ 为一时间序列,其对应的(通常为分段线性的)函数表示为 $f: T \to \mathbb{R}$,其中 $T$ 为时间定义域。基于函数值的滤过主要有以下形式:
\begin{itemize}
    \item \textbf{子水平集滤过 (Sublevel Set Filtration)}: 该方法考察函数值低于某一阈值 $r$ 的区域的拓扑演化。定义子水平集为 $F_r = f^{-1}((-\infty, r]) = \{t \in T \mid f(t) \le r\}$。随着阈值 $r$ 从 $-\infty$ 连续增加至 $+\infty$,这些子水平集形成一族嵌套的拓扑空间序列:$\emptyset = F_{r_0} \subset F_{r_1} \subset \dots \subset F_{r_k} = T$。通过计算该滤过的持久性同调 $PH_*(F_r)$,可以追踪拓扑特征(特别是 $H_0$,对应于低于阈值的连通时间区间)的“诞生”与“消亡” \cite{Chung2020PHApproachTSC}。Chung et al. (2020) 的工作是该方法的一个典型应用,他们基于此滤过计算持久性曲线作为时间序列的拓扑特征 \cite{Chung2020PHApproachTSC}。

    \item \textbf{超水平集/上水平集滤过 (Superlevel Set Filtration)}: 与子水平集滤过对偶,该方法考察函数值高于某一阈值 $r$ 的区域。定义超水平集为 $G_r = f^{-1}([r, +\infty)) = \{t \in T \mid f(t) \ge r\}$。当 $r$ 从 $+\infty$ 减小至 $-\infty$ 时,形成滤过。计算 $PH_*(G_r)$ 可分析函数“波峰”相关的拓扑特征的持续性。Karan \& Kaygun (2021) 在研究中,除了使用 TDE 外,也计算了时间序列子窗口的上、下水平集的 $H_0$ 持久性信息 \cite{Karan2021TSClassificationViaTDA}。
\end{itemize}
此外,也可以先对原始时间序列 $f(t)$ 进行变换,例如计算其导数、积分或进行某种平滑处理,再对变换后的函数应用上述滤过方法。

\subsubsection{与基于点云方法的本质区别}
\label{sec:func_vs_pointcloud}
基于函数的滤过方法与基于点云(如 TDE、值-差分嵌入)的方法在分析目标和信息侧重上存在本质区别:
\begin{itemize}
    \item \textbf{分析空间域不同}: 点云方法旨在通过相空间重构,分析动力系统在(高维)状态空间**中的几何拓扑结构,例如吸引子的形状、维度、连通性等。而函数滤过方法则直接分析时间序列函数在其一维时间定义域上的拓扑特征,这些特征与函数值的起伏、持续时间、阈值穿越行为紧密相关。
    \item \textbf{捕捉特征不同}: 点云方法的持久性特征(如 $H_1, H_2$ 等)通常对应于重构吸引子的环状结构、空洞等几何特征。函数滤过方法(特别是标准的子/超水平集滤过)的持久性特征(主要是 $H_0$)则更多地反映了信号波峰、波谷的数量、持续长度以及它们随阈值变化的合并关系。
    \item \textbf{数据表示依赖}: 函数滤过直接作用于时间序列函数,不依赖于嵌入维度 $d$ 和延迟 $\tau$。点云方法则首先需要构建点云,其结果的有效性依赖于嵌入方法的选择和参数设定。
\end{itemize}

\subsubsection{优点与缺点}
\label{sec:func_filtration_pros_cons}
\textbf{优点}
\begin{itemize}
    \item \textbf{概念与参数简单}: 避免了 TDE 中选择嵌入维度 $d$ 和延迟 $\tau$ 的复杂性和不确定性。
    \item \textbf{直接关联函数值特征}: 能直接、定量地捕捉与信号幅度、峰谷值、持续时间等相关的拓扑信息。
    \item \textbf{计算效率可能较高}: 对于一维函数图,计算其子/超水平集滤过的低维(尤其 $H_0$)持久性同调通常比处理高维点云的 Vietoris-Rips 复形更高效。
    \item \textbf{理论稳定性}: 存在关于函数扰动下持久性图稳定性的理论保证 \cite{Chung2020PHApproachTSC, CohenSteiner2007StabilityPD}。
\end{itemize}

\textbf{缺点}
\begin{itemize}
    \item \textbf{高维信息捕捉受限}: 标准的子/超水平集滤过主要反映 $H_0$ 特征,难以直接揭示复杂动力系统在高维相空间中的精细几何结构(如高维环、扭结等)。
    \item \textbf{解释与动力学关联可能间接}: 其拓扑特征的物理解释与函数值波动直接相关,但可能不如 TDE 重构的吸引子拓扑那样直接反映系统的内在动力学机制。
    \item \textbf{对噪声的敏感性}: 函数值的噪声可能导致产生大量短寿命的 $H_0$ 特征,需要后续处理或使用对噪声鲁棒的 TDA 特征表示。
\end{itemize}

\subsection{其他潜在的时间序列拓扑表示方法}
\label{sec:other_topo_repr}
除了前述基于相空间重构点云和基于函数滤过的主要方法外,还存在其他利用拓扑思想研究时间序列的方法,简述如下:
\begin{itemize}
    \item \textbf{基于图的方法 (Graph-based Methods)}: 此类方法首先将时间序列转化为图(Graph)或网络(Network)结构,例如能见度图 (Visibility Graphs) \cite{Lacasa2008VG} 或序数划分网络 (Ordinal Partition Networks) \cite{McCullough2017OPN}。随后,可以应用图论分析或计算图的持久性同调来提取拓扑特征。这类方法提供了一种不同的视角来捕捉序列中的模式和依赖关系。*(注:相关文献需进一步调研)*
    \item \textbf{多参数持久性同调 (Multiparameter Persistent Homology)}: 传统 PH 仅跟踪单个滤过参数。多参数 PH 允许同时考虑两个或多个参数(例如时间窗口与密度阈值,或函数值与时间导数)。这为捕捉更复杂的依赖关系提供了理论框架,但其计算理论和稳定性分析仍是活跃的研究领域,且计算成本显著更高。Karan \& Kaygun (2021) 也视其为潜在的未来研究方向 \cite{Karan2021TSClassificationViaTDA}。
    \item \textbf{替代嵌入技术 (Alternative Embedding Techniques)}: 理论上,除 TDE 和值-差分嵌入外,也可探索使用其他(非线性)降维技术从原始高维数据(如多变量时间序列)或 TDE 生成的高维点云中创建低维拓扑表示。然而,TDE 因其与动力系统重构理论(Takens 定理)的紧密联系而具有独特的地位和理论支持 \cite{Takens1981DetectingStrangeAttractors}。
\end{itemize}


\subsection{方法比较和深入讨论}
\label{sec:repr_comparison}
本章介绍了将时间序列转化为适用于 TDA 分析的几种主要拓扑表示方法:时间延迟嵌入 (TDE)、值-差分嵌入 (Value-Difference Embedding) 以及基于函数或其变换的滤过。这些方法在理论基础、信息侧重、参数依赖、计算成本和鲁棒性等方面各有特点。

\paragraph{核心对比}
\begin{itemize}
    \item \textbf{理论基础与目标}: TDE 以 Takens 嵌入定理为理论基石 \cite{Takens1981DetectingStrangeAttractors},旨在重构原始动力系统的相空间拓扑结构,尤其适用于分析具有复杂吸引子的非线性或混沌系统 \cite{Umeda2017TSClassificationViaTDA}。值-差分嵌入 (如 Yan et al. (2024) 中所用 \cite{Yan2024PHTSI}) 则直接结合当前值与一阶差分,构建固定低维(二维)表示,侧重于捕捉局部动态变化。基于函数的滤过(如 Chung et al. (2020) \cite{Chung2020PHApproachTSC})直接分析函数值起伏,提取与幅度阈值相关的拓扑特征。
    \item \textbf{参数依赖性}: TDE 需要谨慎选择嵌入维度 $d$ 和时间延迟 $\tau$,参数选择不当会严重影响重构质量 \cite{Karan2021TSClassificationViaTDA}。值-差分嵌入和函数滤过方法避免了这两个参数,但前者固定了输出维度,后者则涉及滤过类型(子/超水平集)的选择及可能的值域处理。
    \item \textbf{信息侧重}: TDE 关注(高维)几何形态;值-差分嵌入关注值与局部变化率;函数滤过关注值与阈值的关系。三者捕捉的时间序列信息维度不同。
    \item \textbf{计算复杂度与鲁棒性}: 函数滤过计算 $H_0$ 通常较快,且有较好的理论稳定性保证 \cite{Chung2020PHApproachTSC}。点云方法(尤其是 TDE)可能生成高维、大规模点云,导致后续 PH 计算(如构建 Vietoris-Rips 复形)成本较高 \cite{Yan2024PHTSI}。不同方法对噪声的敏感度不同,例如差分可能放大噪声,而 Karan \& Kaygun (2021) 提出的子窗口策略旨在降低噪声影响 \cite{Karan2021TSClassificationViaTDA}。Yan et al. (2024) 提出的时间倾斜旨在融入时序信息并适应更多结构 \cite{Yan2024PHTSI}。
\end{itemize}

\paragraph{讨论}
选择何种拓扑表示方法并无绝对最优解,通常取决于具体应用场景、数据特性和分析目标。
\begin{itemize}
    \item 若研究重点是挖掘系统内在的复杂动力学或高维几何结构,TDE 可能是首选。
    \item 若关注信号的局部变化趋势或希望简化嵌入过程,值-差分嵌入提供了一个备选项。
    \item 若分析重点在于信号幅度波动、峰谷特征或基于阈值的事件,基于函数的滤过可能更直接有效。
\end{itemize}
此外,如子窗口、时间倾斜等预处理技术可以与这些表示方法结合,以优化性能或引入额外信息。最终选择的表示方法将直接影响后续章节中特征提取策略的有效性(第 \ref{sec:tde_examples} 章)以及分类模型的构建与性能(第 \ref{chap:classification} 章)。对这些方法的深入理解与恰当选择,是成功运用 TDA 进行时间序列分析的关键一步。
