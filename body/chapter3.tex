\section{时间序列的拓扑表示构建方法}
\subsection{引言:为何需要将时间序列转化为TDA可处理的表示?}
时间序列分析在金融、医疗、工业等领域具有广泛应用,但其非线性、混沌和周期性特性使得传统方法(如统计分析、傅里叶变换)难以捕捉高维拓扑结构或动态特性。例如,脑电图(EEG)信号的复杂时序模式需要更强大的分析工具来提取隐含信息。拓扑数据分析(TDA)通过持久同调(Persistent Homology)提取数据的拓扑特征,如连接组件和孔洞,对噪声鲁棒且能揭示复杂数据的内在模式,特别适合处理非线性时间序列。然而,TDA要求数据以点云或复形形式输入,而时间序列为一维数值序列,无法直接应用。因此,需通过时间延迟嵌入(TDE)等方法将时间序列转化为高维点云,以捕捉动态系统的吸引子或拓扑特性。本章将系统介绍时间序列到TDA表示的转化方法,包括时间延迟嵌入(TDE)、值-差分散入等,分析其数学原理、实现步骤和适用场景,为后续特征提取和分类任务奠定基础。本章结构如下:首先介绍TDE的理论基础和参数选择方法,其次探讨值-差分散入等替代方法,最后比较不同方法的优缺点及适用性。
\subsection{时间延迟嵌入}
\subsubsection{Takens 嵌入定理简介及其在动力系统重构和 TDA 中的应用动机}
TAKENs嵌入定理(Takens' Embedding Theorem)是由荷兰数学家Floris Takens在1981年提出的一个重要数学定理,主要应用于动力系统和时间序列分析领域。它提供了一种从单一标量时间序列数据中重构原始动力系统相空间的方法,尤其在研究混沌系统和非线性动力学时具有深远意义。

在介绍定理之前我们先来介绍两个基本概念:
\begin{itemize}
    \item \textbf{相空间(Phase Space)}: 动力系统的状态空间,描述系统所有可能状态的集合。对于一个n维动力系统,相空间是n维的。
    \item \textbf{吸引子(Attractor)}: 动力系统在长时间演化后趋向的状态或轨迹集合。吸引子可以是点、周期轨道或更复杂的结构,如奇异吸引子。
\end{itemize}
TAKENS嵌入定理的核心思想是,通过对时间序列进行适当的嵌入,可以在高维空间中重构出原始动力系统的相空间结构。具体来说,假设我们有一个一维时间序列$x(t)$,它是一个动力系统在时间上的投影。TAKENS定理指出,如果我们选择合适的嵌入维度$m$和时间延迟$\tau$,那么通过以下方式构造的$m$维向量序列可以近似重构原始动力系统的相空间:
\begin{equation}
    \mathbf{x}_i = (x(t_i), x(t_{i+\tau}), x(t_{i+2\tau}), \ldots, x(t_{i+(m-1)\tau}))
\end{equation}
其中,$i$是时间序列的索引,$t_i$是时间点,$m$是嵌入维度,$\tau$是时间延迟。通过这种方式,我们可以将一维时间序列转化为$m$维向量序列,从而在高维空间中捕捉到原始动力系统的拓扑结构。这个向量构成的重构相空间与原始的$m$维相空间在拓扑上是等价的,即它们具有相同的拓扑特征,如连通性、孔洞等。这一性质使得TAKENS嵌入定理在TDA中具有重要应用,因为我们可以通过分析重构后的点云数据来提取原始动力系统的拓扑特征。
在TDA中,TAKENS嵌入定理的应用主要体现在以下几个方面:
\begin{itemize}
    \item \textbf{重构相空间}: 通过TAKENS嵌入定理,我们可以将一维时间序列转化为高维点云,从而在高维空间中分析数据的拓扑特征。
    \item \textbf{持久同调分析}: 在重构的点云上应用持久同调等拓扑数据分析方法,可以提取数据的拓扑特征,如连通性、孔洞等。这些特征可以用于分类、聚类等任务。
    \item \textbf{噪声鲁棒性}: TAKENS嵌入定理在一定条件下对噪声具有鲁棒性,使得我们可以在存在噪声的情况下仍然能够重构出原始动力系统的拓扑结构。
    \item \textbf{动态系统分析}: 通过对重构的点云进行拓扑分析,可以揭示动态系统的内在结构和行为模式,如周期性、混沌等特性。。
行拓扑特征提取,可以为机器学习模型提供更丰富的特征表示,从而提高分类和回归任务的性能。
    \item \textbf{动态系统建模}: 在建模复杂动态系统时,TAKENS嵌入定理可以帮助我们从观测数据中重构系统的相空间,从而更好地理解系统的行为和特性。
    \item \textbf{数据可视化}: 通过对重构的点云进行可视化,可以帮助我们更直观地理解数据的拓扑结构和内在关系,从而为后续分析提供支持。
    \item \textbf{多尺度分析}: 通过对不同嵌入维度和时间延迟的组合,可以在不同尺度上分析数据的拓扑特征,从而揭示数据的多尺度结构和行为模式。
\end{itemize}


\subsubsection{TDE的核心:从一维时间序列构建高维向量序列(点云)}
TDE 的基本原理是将一维时间序列 \( s(t) \) 通过引入时间延迟 \( \tau \) 和嵌入维度 \( d \),转化为高维向量:
\begin{equation}
\vec{x}(t) = \left[ s(t), s(t - \tau), s(t - 2\tau), \dots, s(t - (d - 1)\tau) \right]
\end{equation}
这些向量构成的集合形成了一个高维点云,即重构的相空间。Takens 定理指出,当嵌入维度 \( d \geq 2m + 1 \)(其中 \( m \) 为原始系统的维数)时,重构的相空间与原始系统的相空间拓扑等价。这种重构使得我们可以从单一观测变量中恢复系统的非线性动力学行为,例如吸引子结构或混沌特性。点云中的每个点代表系统在某时刻的状态,而点云的几何形状则反映了系统的长期演化规律。


\subsubsection{嵌入维度参数\(d\)的选择}
嵌入维度\(d\)决定了重构相空间的维度,直接影响了点云的展开程度.选择适当的d可以捕捉到系统的非线性特征,但过高的m会导致维数灾难,使得数据稀疏,难以提取有效的拓扑特征.常用的方法包括:
\begin{itemize}
    \item \textbf{伪近邻法(FNN)}: 通过计算不同嵌入维度下的假近邻比例来选择合适的嵌入维度。根据 Takens 定理,\( d \geq 2m + 1 \) 可保证拓扑等价性,但在实际应用中,为了降低计算复杂度,通常选择较小的 \( d \),如 \( 2m \) 或更低。
    \item \textbf{互信息法(MI)}: 通过计算时间序列的互信息来确定合适的时间延迟和嵌入维度。具体步骤如下: 1.计算时间序列中不同时间延迟$\tau$下的平均互信息量; 2.选择互信息量最大的时间延迟作为嵌入维度的候选值; 3.通过交叉验证等方法进一步优化嵌入维度。
    \item \textbf{主成分分析(PCA)}: PCA可以通过对时间序列数据的协方差矩阵进行特征值分解来帮助选择嵌入维度。PCA 可以找出数据中的主要成分并判断需要多少维度才能保留数据的大部分变异信息。当主成分的变化趋于平稳时,表明嵌入维度已经足够。这种方法通过分析嵌入空间的结构变化来推断合理的嵌入维度。
    \item \textbf{最小化重构误差}: 通过最小化重构误差来选择嵌入维度。具体方法是将时间序列嵌入到不同的维度中,然后计算重构误差(如均方根误差)并选择最小的嵌入维度。
\end{itemize}

\subsubsection{时间延迟参数$\tau$的选择}


时间延迟嵌入 (TDE) 的效果很大程度上取决于两个关键参数:嵌入维度 $d$ 和时间延迟 $\tau$。选择合适的 $\tau$ 对于重构动力系统的相空间至关重要。其目标在于找到一个 $\tau$ 值,使得嵌入向量的分量 $x(t)$ 和 $x(t+\tau)$ 既不过于相关(避免信息冗余),又不至于完全无关(以保留系统的动态演化信息)。目前,常用的选择 $\tau$ 的方法主要包括:

\begin{enumerate}
    \item \textbf{平均互信息法 (Average Mutual Information, AMI)}:
    互信息 $I(\tau)$ 用于度量 $x(t)$ 和 $x(t+\tau)$ 之间的相互依赖性,能够捕捉线性和非线性关联。其基本思想是寻找一个 $\tau$ 值,使得 $x(t+\tau)$ 相对于 $x(t)$ 能够提供尽可能多的“新”信息,但同时两者之间仍然存在足够的依赖关系以反映系统的动力学。
    \begin{itemize}
        \item \textbf{选择标准}: 通常选择计算得到的互信息函数 $I(\tau)$ 随 $\tau$ 变化的曲线上的\textbf{第一个局部最小值}对应的 $\tau$ 作为最优延迟。这个最小值点被认为是在减少冗余和保留动态之间的一个良好平衡点。
        \item \textbf{优点}: 能够有效处理非线性系统,是目前较为推荐和常用的方法。
    \end{itemize}

    \item \textbf{自相关函数法 (Autocorrelation Function, ACF)}:
    自相关函数衡量时间序列与其自身在不同延迟 $\tau$ 下的线性相关性。
    \begin{itemize}
        \item \textbf{选择标准}: 通常选择自相关函数值\textbf{第一次下降到零}或者某个特定阈值(例如 $1/e$)时的 $\tau$ 值。这表示在该延迟下,序列与其滞后版本的线性相关性较弱。
        \item \textbf{局限性}: 主要捕捉线性相关性,对于具有复杂非线性动力学的时间序列,可能不是最佳选择。
    \end{itemize}

    \item \textbf{其他方法与考虑因素}:
    \begin{itemize}
        \item \textbf{几何/启发式方法}: 对于周期或准周期信号,$\tau$ 可以根据信号的特征时间尺度(如周期的某个分数)来选择。
        \item \textbf{与嵌入维度 $d$ 的关联}: 有些研究指出最优的 $\tau$ 和 $d$ 可能存在关联,尤其是在处理有噪声或有限长度的数据时,可能需要联合考虑这两个参数。
    \end{itemize}
\end{enumerate}

\textbf{总结}: 平均互信息法 (AMI) 因其能捕捉非线性依赖关系,通常被认为是选择时间延迟 $\tau$ 的首选方法。然而,最佳选择可能仍需根据具体时间序列数据的特性(如噪声、长度、周期性)以及研究目标进行调整和验证。

% 注意:在你的论文中,你可能需要为这些方法添加参考文献,
% 例如引用 Fraser & Swinney (1986) 关于互信息法的论文,
% 或其他关于吸引子重构和参数选择的综述性文献或教科书。
\subsubsection{文献中 TDE 应用实例}
\label{sec:tde_examples}

时间延迟嵌入 (TDE) 作为一种从时间序列重构系统相空间的重要技术,在拓扑数据分析 (TDA) 的诸多研究中得到了广泛应用。以下列举几例说明 TDE 在不同场景下的具体应用:

\paragraph{揭示周期与混沌信号的内在结构}
Umeda (2017) 在其研究中应用 TDE (固定嵌入维度 $p=3$, 时间延迟 $\tau=1$) 从多种合成时间序列生成拟吸引子 (quasi-attractors) \cite{Umeda2017TSClassificationViaTDA}。其实验结果(如图 6(b) 所示)清晰地展示了:
\begin{itemize}
    \item 具有不同幅度和周期的正弦波信号,通过 TDE 嵌入后形成了不同半径和角度的环状或椭圆结构,直观地反映了原始信号的周期性和幅度信息 \cite{Umeda2017TSClassificationViaTDA}。
    \item 对于来自 Logistic 映射的混沌时间序列,尽管它们的初始条件不同导致波形差异巨大,但经过 TDE 嵌入后,它们都呈现出相似的抛物线状吸引子结构 \cite{Umeda2017TSClassificationViaTDA}。
\end{itemize}
这表明 TDE 能够有效地捕捉时间序列背后的动力学规律(如周期性、混沌吸引子),而不受表面波形变化的影响。

\paragraph{区分不同动态系统的嵌入形态}
Karan 和 Kaygun (2021) 也通过一个实例展示了 TDE 对系统动态和嵌入参数的敏感性 \cite{Karan2021TSClassificationViaTDA}。他们对添加了噪声的正弦信号 $y = \sin x$ 和 $y = \sin^5 x$ 进行 TDE。结果(如图 4 所示)表明:
\begin{itemize}
    \item 当嵌入维度较低 ($d=15$) 时,两种信号的嵌入点云呈现出不同的几何形状(前者近似椭圆,后者类似眼镜形)\cite{Karan2021TSClassificationViaTDA}。
    \item 当嵌入维度较高 ($d=50$) 时,两者的嵌入点云都趋向于圆形结构,但半径可能不同 \cite{Karan2021TSClassificationViaTDA}。
\end{itemize}
这个例子说明了 TDE 能够区分具有不同非线性特征的系统,并且嵌入维度 $d$ 的选择会显著影响重构相空间的几何形态。

\paragraph{作为 TDA 流程的预处理步骤}
在许多将 TDA 应用于时间序列分类的研究中,TDE 被用作核心的预处理步骤,将原始序列转化为适合计算持久性同调的点云数据。
\begin{itemize}
    \item Karan 和 Kaygun (2021) 在处理 WESAD 和 DriveDB 数据集中的生理信号(如 BVP, ECG)时,先将信号分割成较短的子窗口(例如 4 秒),然后对每个子窗口应用 TDE(尝试了多种与采样频率 $f_s$ 相关的嵌入维度,如 $d=0.5f_s, f_s, 1.5f_s, 2f_s$),最后在生成的点云上计算持久性同调以提取拓扑特征用于压力状态识别 \cite{Karan2021TSClassificationViaTDA}。
    \item Umeda (2017) 在进行人体活动识别任务时,也是先将来自可穿戴传感器(陀螺仪、EEG、EMG)的时间序列通过 TDE 转换为拟吸引子点云,然后计算其 Betti 序列,并最终输入到 CNN 模型中进行分类 \cite{Umeda2017TSClassificationViaTDA}。
\end{itemize}
这些实例表明,TDE 是连接原始时间序列数据和后续拓扑特征提取(如持久性同调计算)的关键桥梁。

总而言之,文献中的这些实例展示了 TDE 在揭示时间序列内在动力学结构(周期性、混沌吸引子等)、区分不同系统以及作为 TDA 分析流程标准预处理步骤方面的重要作用和广泛应用。


\subsubsection{优点与缺点}

\subsection{值-差分嵌入}
值-差分嵌入(Value-Difference Embedding)是一种将时间序列转化为高维点云的替代方法。与传统的时间延迟嵌入方法不同,值-差分嵌入不仅考虑了时间序列的当前值,还考虑了相邻时间点之间的差异。这种方法可以更好地捕捉时间序列中的动态变化和局部结构。

\subsubsection{方法原理}
在这篇论文 \cite{Yan2024PHTSI} 中,值差分嵌入特指将一个单变量时间序列 \( X = (x_1, x_2, \dots, x_n) \) 转换为一个二维点集(点云)。转换方式是生成一系列点 \( (x_i, x_{i+1} - x_i) \),其中 \( i \) 从 1 变化到 \( n-1 \)。得到的点云 \( P \) 就是 \( \{ (x_i, x_{i+1} - x_i) \mid i = 1, \dots, n-1 \} \) [cite: 398, 418]。

\subsubsection{动机}
提出这种特定二维嵌入方法的主要动机包括:
\begin{itemize}
    \item \textbf{捕捉动态变化}: 该方法旨在通过同时包含时间序列的当前值 ($x_i$) 和其一阶差分 ($x_{i+1} - x_i$),来更好地展现时间序列在周期内和周期之间的动态变化特征 [cite: 398, 415]。差分项直接反映了序列的局部变化趋势。
    \item \textbf{简化嵌入过程}: 相比于传统的时间延迟嵌入 (TDE) 需要仔细选择嵌入维度 $d$ 和时间延迟 $\tau$ (这些参数的选择对重构效果影响很大,且在不同数据集上标准不一,计算复杂 [cite: 415]),这种值-差分嵌入直接将一维时间序列映射到固定的二维空间,避免了参数选择的困难。
    \item \textbf{为拓扑分析提供基础}: 构建的二维点云可以直接作为拓扑数据分析(如计算持续同调)的输入,以提取时间序列数据中蕴含的拓扑结构特征 [cite: 398, 415]。
\end{itemize}

\subsubsection{优点与缺点}
基于其原理和动机,可以总结该方法的优缺点如下:

\textbf{优点:}
\begin{itemize}
    \item \textbf{实现简单}: 将一维时间序列映射到二维点云的过程直观且易于实现,不需要复杂的参数调优 [cite: 415]。
    \item \textbf{捕捉局部动态}: 通过引入差分项,能够直接反映时间序列的增减变化和速率信息,有助于捕捉局部动态特征 [cite: 398]。
    \item \textbf{固定低维输出}: 始终生成二维点云,可能使得后续的持续同调计算(尤其是在点云规模不大时)相对高效 [cite: 398]。
\end{itemize}

\textbf{缺点:}
\begin{itemize}
    \item \textbf{维度限制}: 将数据固定嵌入到二维空间,可能无法完全“展开”高维动态系统真实的吸引子结构。根据Takens定理,对于复杂系统,可能需要更高的嵌入维度才能保证拓扑等价性。这种二维嵌入可能丢失部分高维拓扑信息。
    \item \textbf{对噪声敏感性 (可能)}: 差分操作 ($x_{i+1} - x_i$) 可能会放大原始时间序列中存在的噪声,从而影响点云的结构和后续拓扑特征的稳定性。 (注:这一点是基于差分操作特性的推断,原文未明确强调)
    \item \textbf{信息整合有限}: 只考虑了当前值和紧邻的下一个值之间的关系,可能不如TDE(可以设置更大的 $\tau$)那样灵活地捕捉不同时间尺度上的依赖关系。
\end{itemize}

\subsection{基于时间序列函数或其变换的滤过}
\label{sec:func_filtration}
除了通过相空间重构构建点云,拓扑数据分析(TDA)亦可直接应用于时间序列函数本身或其变换。此类方法通过在函数的值域上定义滤过(filtration),进而利用持久性同调(persistent homology, PH)来捕捉与函数值演化相关的拓扑特征。
\subsubsection{方法原理}
\label{sec:func_filtration_principle}

令 $s(t)$ 为一时间序列,其对应的(通常为分段线性的)函数表示为 $f: T \to \mathbb{R}$,其中 $T$ 为时间定义域。基于函数值的滤过主要有以下形式:
\begin{itemize}
    \item \textbf{子水平集滤过 (Sublevel Set Filtration)}: 该方法考察函数值低于某一阈值 $r$ 的区域的拓扑演化。定义子水平集为 $F_r = f^{-1}((-\infty, r]) = \{t \in T \mid f(t) \le r\}$。随着阈值 $r$ 从 $-\infty$ 连续增加至 $+\infty$,这些子水平集形成一族嵌套的拓扑空间序列:$\emptyset = F_{r_0} \subset F_{r_1} \subset \dots \subset F_{r_k} = T$。通过计算该滤过的持久性同调 $PH_*(F_r)$,可以追踪拓扑特征(特别是 $H_0$,对应于低于阈值的连通时间区间)的“诞生”与“消亡” \cite{Chung2020PHApproachTSC}。Chung et al. (2020) 的工作是该方法的一个典型应用,他们基于此滤过计算持久性曲线作为时间序列的拓扑特征 \cite{Chung2020PHApproachTSC}。

    \item \textbf{超水平集/上水平集滤过 (Superlevel Set Filtration)}: 与子水平集滤过对偶,该方法考察函数值高于某一阈值 $r$ 的区域。定义超水平集为 $G_r = f^{-1}([r, +\infty)) = \{t \in T \mid f(t) \ge r\}$。当 $r$ 从 $+\infty$ 减小至 $-\infty$ 时,形成滤过。计算 $PH_*(G_r)$ 可分析函数“波峰”相关的拓扑特征的持续性。Karan \& Kaygun (2021) 在研究中,除了使用 TDE 外,也计算了时间序列子窗口的上、下水平集的 $H_0$ 持久性信息 \cite{Karan2021TSClassificationViaTDA}。
\end{itemize}
此外,也可以先对原始时间序列 $f(t)$ 进行变换,例如计算其导数、积分或进行某种平滑处理,再对变换后的函数应用上述滤过方法。

\subsubsection{与基于点云方法的本质区别}
\label{sec:func_vs_pointcloud}
基于函数的滤过方法与基于点云(如 TDE、值-差分嵌入)的方法在分析目标和信息侧重上存在本质区别:
\begin{itemize}
    \item \textbf{分析空间域不同}: 点云方法旨在通过相空间重构,分析动力系统在(高维)状态空间**中的几何拓扑结构,例如吸引子的形状、维度、连通性等。而函数滤过方法则直接分析时间序列函数在其一维时间定义域上的拓扑特征,这些特征与函数值的起伏、持续时间、阈值穿越行为紧密相关。
    \item \textbf{捕捉特征不同}: 点云方法的持久性特征(如 $H_1, H_2$ 等)通常对应于重构吸引子的环状结构、空洞等几何特征。函数滤过方法(特别是标准的子/超水平集滤过)的持久性特征(主要是 $H_0$)则更多地反映了信号波峰、波谷的数量、持续长度以及它们随阈值变化的合并关系。
    \item \textbf{数据表示依赖}: 函数滤过直接作用于时间序列函数,不依赖于嵌入维度 $d$ 和延迟 $\tau$。点云方法则首先需要构建点云,其结果的有效性依赖于嵌入方法的选择和参数设定。
\end{itemize}

\subsubsection{优点与缺点}
\label{sec:func_filtration_pros_cons}
\textbf{优点}
\begin{itemize}
    \item \textbf{概念与参数简单}: 避免了 TDE 中选择嵌入维度 $d$ 和延迟 $\tau$ 的复杂性和不确定性。
    \item \textbf{直接关联函数值特征}: 能直接、定量地捕捉与信号幅度、峰谷值、持续时间等相关的拓扑信息。
    \item \textbf{计算效率可能较高}: 对于一维函数图,计算其子/超水平集滤过的低维(尤其 $H_0$)持久性同调通常比处理高维点云的 Vietoris-Rips 复形更高效。
    \item \textbf{理论稳定性}: 存在关于函数扰动下持久性图稳定性的理论保证 \cite{Chung2020PHApproachTSC, CohenSteiner2007StabilityPD}。
\end{itemize}

\textbf{缺点}
\begin{itemize}
    \item \textbf{高维信息捕捉受限}: 标准的子/超水平集滤过主要反映 $H_0$ 特征,难以直接揭示复杂动力系统在高维相空间中的精细几何结构(如高维环、扭结等)。
    \item \textbf{解释与动力学关联可能间接}: 其拓扑特征的物理解释与函数值波动直接相关,但可能不如 TDE 重构的吸引子拓扑那样直接反映系统的内在动力学机制。
    \item \textbf{对噪声的敏感性}: 函数值的噪声可能导致产生大量短寿命的 $H_0$ 特征,需要后续处理或使用对噪声鲁棒的 TDA 特征表示。
\end{itemize}

\subsection{其他潜在的时间序列拓扑表示方法}
\label{sec:other_topo_repr}
除了前述基于相空间重构点云和基于函数滤过的主要方法外,还存在其他利用拓扑思想研究时间序列的方法,简述如下:
\begin{itemize}
    \item \textbf{基于图的方法 (Graph-based Methods)}: 此类方法首先将时间序列转化为图(Graph)或网络(Network)结构,例如能见度图 (Visibility Graphs) \cite{Lacasa2008VG} 或序数划分网络 (Ordinal Partition Networks) \cite{McCullough2017OPN}。随后,可以应用图论分析或计算图的持久性同调来提取拓扑特征。这类方法提供了一种不同的视角来捕捉序列中的模式和依赖关系。*(注:相关文献需进一步调研)*
    \item \textbf{多参数持久性同调 (Multiparameter Persistent Homology)}: 传统 PH 仅跟踪单个滤过参数。多参数 PH 允许同时考虑两个或多个参数(例如时间窗口与密度阈值,或函数值与时间导数)。这为捕捉更复杂的依赖关系提供了理论框架,但其计算理论和稳定性分析仍是活跃的研究领域,且计算成本显著更高。Karan \& Kaygun (2021) 也视其为潜在的未来研究方向 \cite{Karan2021TSClassificationViaTDA}。
    \item \textbf{替代嵌入技术 (Alternative Embedding Techniques)}: 理论上,除 TDE 和值-差分嵌入外,也可探索使用其他(非线性)降维技术从原始高维数据(如多变量时间序列)或 TDE 生成的高维点云中创建低维拓扑表示。然而,TDE 因其与动力系统重构理论(Takens 定理)的紧密联系而具有独特的地位和理论支持 \cite{Takens1981DetectingStrangeAttractors}。
\end{itemize}


\subsection{方法比较和深入讨论}
\label{sec:repr_comparison}
本章介绍了将时间序列转化为适用于 TDA 分析的几种主要拓扑表示方法:时间延迟嵌入 (TDE)、值-差分嵌入 (Value-Difference Embedding) 以及基于函数或其变换的滤过。这些方法在理论基础、信息侧重、参数依赖、计算成本和鲁棒性等方面各有特点。

\paragraph{核心对比}
\begin{itemize}
    \item \textbf{理论基础与目标}: TDE 以 Takens 嵌入定理为理论基石 \cite{Takens1981DetectingStrangeAttractors},旨在重构原始动力系统的相空间拓扑结构,尤其适用于分析具有复杂吸引子的非线性或混沌系统 \cite{Umeda2017TSClassificationViaTDA}。值-差分嵌入 (如 Yan et al. (2024) 中所用 \cite{Yan2024PHTSI}) 则直接结合当前值与一阶差分,构建固定低维(二维)表示,侧重于捕捉局部动态变化。基于函数的滤过(如 Chung et al. (2020) \cite{Chung2020PHApproachTSC})直接分析函数值起伏,提取与幅度阈值相关的拓扑特征。
    \item \textbf{参数依赖性}: TDE 需要谨慎选择嵌入维度 $d$ 和时间延迟 $\tau$,参数选择不当会严重影响重构质量 \cite{Karan2021TSClassificationViaTDA}。值-差分嵌入和函数滤过方法避免了这两个参数,但前者固定了输出维度,后者则涉及滤过类型(子/超水平集)的选择及可能的值域处理。
    \item \textbf{信息侧重}: TDE 关注(高维)几何形态;值-差分嵌入关注值与局部变化率;函数滤过关注值与阈值的关系。三者捕捉的时间序列信息维度不同。
    \item \textbf{计算复杂度与鲁棒性}: 函数滤过计算 $H_0$ 通常较快,且有较好的理论稳定性保证 \cite{Chung2020PHApproachTSC}。点云方法(尤其是 TDE)可能生成高维、大规模点云,导致后续 PH 计算(如构建 Vietoris-Rips 复形)成本较高 \cite{Yan2024PHTSI}。不同方法对噪声的敏感度不同,例如差分可能放大噪声,而 Karan \& Kaygun (2021) 提出的子窗口策略旨在降低噪声影响 \cite{Karan2021TSClassificationViaTDA}。Yan et al. (2024) 提出的时间倾斜旨在融入时序信息并适应更多结构 \cite{Yan2024PHTSI}。
\end{itemize}

\paragraph{讨论}
选择何种拓扑表示方法并无绝对最优解,通常取决于具体应用场景、数据特性和分析目标。
\begin{itemize}
    \item 若研究重点是挖掘系统内在的复杂动力学或高维几何结构,TDE 可能是首选。
    \item 若关注信号的局部变化趋势或希望简化嵌入过程,值-差分嵌入提供了一个备选项。
    \item 若分析重点在于信号幅度波动、峰谷特征或基于阈值的事件,基于函数的滤过可能更直接有效。
\end{itemize}
此外,如子窗口、时间倾斜等预处理技术可以与这些表示方法结合,以优化性能或引入额外信息。最终选择的表示方法将直接影响后续章节中特征提取策略的有效性(第 \ref{sec:tde_examples} 章)以及分类模型的构建与性能(第 \ref{chap:classification} 章)。对这些方法的深入理解与恰当选择,是成功运用 TDA 进行时间序列分析的关键一步。
